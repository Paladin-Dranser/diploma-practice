\section{Тэхнічна-эканамічнае абгрунтаванне}

\subsection{Кароткая характарыстыка распрацаванага праграмнага забеспячэння}

Дыпломны праект прадстаўляе сабой праграмнае мадэліраванне гібрыднай SDN сеткі на базе EIGRP пратакола. У сетку ўводзіцца дадатковы элемент-кантролер, які ажыццяўляе маніторынг характарыстык сеткавых
элементаў і прыняцця рашэння пра пераразлік метрыкі і перабудовы сеткавай тапалогіі на маршрутызатары.

Асноўныя перавагі пабудовы гібрыднай SDN мадэлі:
\begin{enumerate}
    \item працэсы перадачы і кіраванне данымі падзелены. Гэта значыць, што маршрутызатары
        адказваюць толькі за функцыю перадачы даных, у той час як за логіку маршрутызацыі
        адказвае адмысловы кантролер;
    \item кіраванне сеткай цэнтралізавана пры дапамозе ўніфіцыраваных праграмных сродкаў.
        Так як кіраванне сеткай ажыццяўляецца пры дапамозе знешніх праграмных сродкаў, гэта
        дазваляе лёгка ўносіць неабходныя настройкі на ўсе ўстройства ў сетцы.
        Пры гэтым неабходна дадаць, што дадзены прынцып дазваляе наблізіць канфігурацыю
        сеткавых устройств да канцэпцыі <<інфраструктура як код>>, што забяспечвае
        аўтаматызацыю канфігурацыі сеткі;
    \item эканомія на сеткавых устройствах. Так як кіруючая функцыя рэалізаваная на асобным
        кантролеры, то сеткавае ўстройства больш не патрабуе наяўнасці вялікіх рэсурсаў
        (працэсар, памяць і да т. п.), што спрашчае яго структуру, а значыць і кошт устройства;
    \item памяншэнне працазатрат на суправаджэнне сеткі. Так як ўся інфармацыя і канфігурацыя
        маршрутызатараў знаходзіцца ў адным месцы, то супрацоўнікі патрабуецца менш часу для
        карэктыроўкі канфігурацыі пры неабходнасці;
    \item павелічэнне эфектыўнасці выкарыстання рэсурсаў сеткі. Кантролер дазваляе адсочваць
        характарыстыкі ўстройств у рэльным часе, што дазваляе аўтаматычна змяняць канфігурацыю
        ўстройства для дасягнення аптымальнай канфігурацыі ў кожны момант часу.
\end{enumerate}

\subsection{Разлік сабекошту і цаны праграмнага прадукту}

Першым крокам неабходна вызначыць вартасную ацэнку распрацаванага праграмнага прадукту. Яна прадугледжвае складанне каштарысу выдаткаў, на падставе якога вызначаецца сабекошт і адпускная цана распрацаванага праграмнага прадукту.
Каштарыс выдаткаў ўключае наступныя выдаткі:
\begin{enumerate}
    \item выдаткі на аплату працы выканаўцаў;
    \item адлічэнні на сацыяльныя патрэбы;
    \item матэрыялы і камплектуючыя;
    \item выдаткі на навуковыя камандзіроўкі;
    \item іншыя прамыя выдаткі;
    \item накладныя выдаткі.
\end{enumerate}

У артыкул «Выдаткі на аплату працы» ўключаюцца асноўная і дадатковая заработная плата ўсіх супрацоўнікаў, занятых распрацоўкай праграмнага забеспячэння для SDN кантролера. У дадзеным выпадку напісаннем праграмы займаецца адзін інжынер-праграміст.

Разлік асноўнай заработнай платы выконваецца па формуле (\ref{eqn: eqn1}):
\begin{equation}
    \label{eqn: eqn1}
    \text{З}_\text{o} = \sum_{i=1}^{n} \text{T}_\text{di} \cdot \text{Ф}_\text{П} \cdot \text{K}_\text{ПР}
\end{equation}
\begin{Explanation}
    \item[дзе] $\text{T}_\text{di}$ -- дзённая тарыфная стаўка $i$-га выканаўца, руб;
    \item $\text{Ф}_\text{п}$ -- планавы фонд працоўнага часу - час, у перыяд якога
        супрацоўнік $i$-ай катэгорыі прымаў удзел у распрацоўцы, дзён;
    \item $\text{К}_\text{ПР}$ -- каэфіцыент прэміравання за выкананне планавых
        паказальнікаў работы, роўны 1,4.
\end{Explanation}

Дзённая тарыфная стаўка $i$-га выканаўца разлічваецца наступным
чынам:
\begin{equation}
    \text{T}_\text{дi} = \text{T}_\text{мі} / \text{Ф}_\text{р}
\end{equation}
\begin{Explanation}
    \item[дзе] $\text{T}_\text{мі}$ -- месячная тарыфная стаўка $i$-га выканаўца, руб;
    \item $\text{Ф}_\text{р}$ -- сярэднемесячны фонд працоўна часу, дні.
\end{Explanation}

Разліковая норма працоўнага часу ў днях на 2020 год пры пяцідзённым працоўным тыдні складае 256 дзён. Тады сярэдне месячны фонд працоўнага часу складае прыблізна 21 дзень.

Месячная тарыфная стаўка кожнага выканаўца вызначаецца па
формуле (\ref{enq: enq3}):
\begin{equation}
    \label{enq: enq3}
    \text{T}_\text{M} = \text{T}_\text{М1p} \cdot \text{T}_\text{K}
\end{equation}
\begin{Explanation}
    \item[дзе] $\text{T}_\text{M1p}$ -- дзеючая месячная тарыфная стаўка першага разраду (82 рубелі);
    \item $\text{T}_\text{K}$ -- тарыфны каэфіцыент (1,21).
\end{Explanation}

Дадатковая заработная плата персанала ўключае выплаты, якія
прадугледжаны дзеючым заканадаўствам за непрапрацаваны час --
водпуск, выплата за выслугу гадоў і да таго падобнае.
Дадатковая заработная плата вызначаецца па формуле (\ref{eqn: eqn4}):

\begin{equation}
    \label{eqn: eqn4}
    \text{З}_\text{Д} = \text{З}_\text{o} \cdot \text{H}_\text{Д} / 100\%
\end{equation}
\begin{Explanation}
    \item[дзе] $\text{H}_\text{Д}$ -- нарматыў дадатковай заработнай платы (10\%).
\end{Explanation}

Агульныя затраты на аплату працы (ЗАП) вызначаецца па формуле
(\ref{eqn: eqn5}):
\begin{equation}
    \label{eqn: eqn5}
    \text{ЗАП} = \text{З}_\text{o} + \text{З}_\text{Д}
\end{equation}

Разлік асноўнай заработнай платы персаналу прадстаўлены ў табліцы
\ref{table: 5.1}.

\begin{table}[htp]
    \caption{Разлік асноўнай заработнай платы навукова-вытворчага персаналу}
    \begin{tabularx}{\textwidth}{ 
        | p{2.1cm}
        | >{\centering\arraybackslash}X 
        | >{\centering\arraybackslash}X 
        | >{\centering\arraybackslash}X 
        | >{\centering\arraybackslash}X 
        | >{\centering\arraybackslash}X 
        | >{\centering\arraybackslash}X 
        | >{\centering\arraybackslash}X 
        | >{\centering\arraybackslash}X |
    }
    \hline
        Катэгорыя супрацоўнікаў
        & Колькасць чалавек
        & Тарыфны каэфіцыент
        & Месячная тарыфная стаўка
        & Дзённая тарыфная стаўка
        & Планавы фонд працоўнага часу аднаго супрацоўніка
        & Аплата за адпрацаваны час аднаго супрацоўніка
        & Асноўная заработная плата аднаго супрацоўніка з улікам прэміі
        & Асноўная заработная плата з улікам прэміі, усяго \\
    \hline
        Інжынер-праграміст
        & 1
        & 1,21
        & 99,22
        & 4,72
        & 21
        & 99,12
        & 99,12
        & 109,03  \\
    \hline
        \multicolumn{8}{|l|}{Усяго}
        & 109,03 \\
    \hline
    \end{tabularx}
    \label{table: 5.1}
\end{table}

Адлічэнні на сацыяльныя патрэбы (АСП) уключаюць адлічэнні ў фонд
сацыяльнай абароны і адлічэнні ў Белдзяржстрах на страхаванне ад
няшчасных выпадкаў і прафейсійных захворванняў.

Адлічэнне на сацыяльныя патрэбы вылічваюцца па формуле
(\ref{eqn: eqn6}):

\begin{equation}
    \label{eqn: eqn6}
    \text{АСП} = \text{ФСАН} + \text{БДС}
\end{equation}
\begin{Explanation}
    \item[дзе] $\text{ФСАН}$ -- адлічэнні ў фонд сацыяльнай абароны насельніцтва, руб;
    \item $\text{БДС}$ -- адлічэнне ў Белдзяржстрах на страхаванне ад няшчасных выпадкаў і прафейсійных захворванняў, руб.
\end{Explanation}

Сума адлічэнняў у фонд сацыяльнай абароны насельніцтва вызначаецца
ў працэнтаў ад сумы асноўнай і дадатковай заработнай платы ўсіх катэгорый супрацоўнікаў, якія маюць дачыненне да выканання распрацоўкі праграмнага забеспячэння.

Адлічэнне ў фонд сацыяльнай абароны насельніцтва вызначаюцца па формуле (\ref{eqn: eqn7}):

\begin{equation}
    \label{eqn: eqn7}
    \text{ФСАН} = \frac{\text{ЗАП} \cdot \text{S}_\text{ФСАН}}{100}
\end{equation}
\begin{Explanation}
    \item[дзе] $\text{S}_\text{ФСАН}$ -- стаўка адлічэнняў у фонд сацыяльнай абароны насельніцтва (34\%).
\end{Explanation}

Адлічэнні ў Белдзяржстрах на страхаванне ад няшчаных выпадкаў і прафесійных захворванняў разлічваецца па формуле (\ref{eqn: eqn8}):

\begin{equation}
    \label{eqn: eqn8}
    \text{БДС} = \frac{\text{ЗАП} \cdot \text{H}_\text{БДС}}{100}
\end{equation}
\begin{Explanation}
    \item[дзе] $\text{H}_\text{БДС}$ -- нарматыў адлічэнняў у Белдзяржстрах, устаноўлены для арганізацыі (0,6\%).
\end{Explanation}

Выдаткі па артыкулу <<Матэрыялы і складнікі вырабу>> (М) вызначаюцца
на падстаце каштарысу выдаткаў, якая распрацоўваецца на праграмны
прадукт з улікам дзеючых нарматываў. Па дадзенаму артыкулу выдаткаў
адлюстроўваюцца выдаткі на знешнія прылады памяці, картрыджы, паперу і да таго падобнага. Сума выдаткаў на расходныя матэрыялы,
неабходныя для распрацоўкі прадукту, вызначаюцца паводле нарматыву
ў працэнтах да фонду заработнай платы распрацоўшчыкаў, які ўстанаўліваецца арганізацыяй на ўзроўні 3-5\%.

Выдаткі па артыкулу <<Матэрыялы і складнікі вырабу>> разлічваецца
па формуле (\ref{eqn: eqn9}):

\begin{equation}
    \label{eqn: eqn9}
    \text{M} = \frac{\text{З}_\text{o} \cdot \text{H}_\text{МЗ}}{100}
\end{equation}
\begin{Explanation}
    \item[дзе] $\text{H}_\text{МЗ}$ -- норма расходу матэрыялаў ад асноўнай заработнай платы (3\%).
\end{Explanation}

Расход па артыкулу <<Навуковыя камандзіроўкі>> вызначаецца па нарматыву расходу на камандзіроўкі ў цэлым на арганізацыю да асноўнай заработнай платы ў працэнтах.

Расход па артыкулу <<Навуковыя камандзіроўкі>> ($\text{P}_\text{НК}$) вызначаецца па формуле (\ref{eqn: eqn10}):

\begin{equation}
    \label{eqn: eqn10}
    \text{P}_\text{НК} = \frac{\text{З}_\text{o} \cdot \text{H}_\text{РНК}}{100}
\end{equation}
\begin{Explanation}
    \item[дзе] $\text{H}_\text{РНК}$ -- нарматыў расходу на камандзіроўкі ў цэлам на арганізацыю (10\%).
\end{Explanation}

Расход па артыкулу <<Іншыя прамыя выдаткі>> $\text{ІВ}$ на праграмнае забеспячэнне ўключае выдаткі на набыццё і падрыхтоўку спецыяльнай навукова-тэхнічнай інфармацыі і спецыяльнай літаратуры.

Іншыя прамыя выдаткі разлічваюцца па формуле (\ref{eqn: eqn11}):

\begin{equation}
    \label{eqn: eqn11}
    \text{ІВ} = \frac{\text{З}_\text{o} \cdot \text{H}_\text{ІВ}}{100}
\end{equation}
\begin{Explanation}
    \item[дзе] $\text{H}_\text{ІВ}$ -- нарматыў іншых выдаткаў у цэлым на арганізацыю (20\%).
\end{Explanation}

Расход па артыкулу <<Накладныя выдаткі>>, звязаныя з неабходнасцю
выдаткаў на агульнагаспадарчыя патрэбы, адносяцца на канкрэтны праграмны прадукт
паводле нарматыву накладных выдаткаў ў працэтных адносінах да асноўнай заработнай платы выканаўцаў. Нарматыў складаецца ў цэлым на арганізацыю і складае 10\%.

Выдаткі па артыкулу <<Накладныя выдаткі>> разлічваецца па формуле (\ref{eqn: eqn12}):

\begin{equation}
    \label{eqn: eqn12}
    \text{P}_\text{H} = \frac{\text{З}_\text{o} \cdot \text{H}_\text{РН}}{100}
\end{equation}
\begin{Explanation}
    \item[дзе] $\text{Р}_\text{H}$ -- накладныя выдаткі на канкрэтны праграмны прадукт, руб;
    \item $\text{H}_\text{РН}$ -- нарматыў накладных выдаткаў у цэлым на арганізацыю (10\%).
\end{Explanation}

Агульная сума выдаткаў ($\text{C}_\text{П}$) па ўсіх артыкулах каштарысу на праграмны прадукт разлічваецца па формуле (\ref{eqn: eqn13}):

\begin{equation}
    \label{eqn: eqn13}
    \text{C}_\text{П} = \text{ЗАП} + \text{АСП} + \text{M}
    + \text{Р}_\text{НК} + \text{ІВ} + \text{P}_\text{Н}
\end{equation}

Прагназуемая цана праграмнага прадукту ($\text{Ц}_\text{АДП}$) разлічваецца па формуле (\ref{eqn: eqn14}):

\begin{equation}
    \label{eqn: eqn14}
    \text{Ц}_\text{АДП} = \text{C}_\text{П} + \text{П} + \text{ПДК}
\end{equation}
\begin{Explanation}
    \item[дзе] $\text{П}$ -- планавы прыбытак, руб;
    \item ПДК -- падатак на дадатковы кошт, руб.
\end{Explanation}

Прыбытак (П) неабходна вызначыць з запланаванай нормы рэнтабельнасці.

Прыбытак разлічваецца па формуле (\ref{eqn: eqn15}):

\begin{equation}
    \label{eqn: eqn15}
    \text{П} = \frac{\text{C}_\text{П} \cdot \text{P}_\text{П}}{100}
\end{equation}
\begin{Explanation}
    \item[дзе] $\text{P}_\text{П}$ -- узровень рэнтабельнасці (15\%).
\end{Explanation}

Падатак на дадатковы кошт (ПДК), які ўключаецца ў цану праграмнага прадукту, разлічваецца па формуле (\ref{eqn: eqn16}):

\begin{equation}
    \label{eqn: eqn16}
    \text{НДС} = \frac{\text{C}_\text{П} \cdot \text{S}_\text{ПДК}}{100}
\end{equation}
\begin{Explanation}
    \item[дзе] $\text{S}_\text{ПДК}$ -- стаўка падатку на дадатковы кошт (20\%).
\end{Explanation}

Арганізацыя-распрацоўшчык удзельнічае ў асваенні праграмнага прадукту і нясе адпаведныя выдаткі, на якія складаецца каштарыс, які аплачваецца заказчыкам паводле дагавору. Каштарысам прадугледжваецца не толькі выдаткі, аднак і падаткі, які прадугледжваюцца заканадаўствам, і прыбытак арганізацыі-распрацоўшчыка.

Выдаткі на асваенне вызначаюцца па формуле (\ref{eqn: eqn17}):

\begin{equation}
    \label{eqn: eqn17}
    \text{P}_\text{а} = \frac{\text{C}_\text{П} \cdot \text{H}_\text{a}}{100}
\end{equation}
\begin{Explanation}
    \item[дзе] $\text{P}_\text{a}$ -- выдаткі на асваенне;
    \item $\text{H}_\text{a}$ -- нарматыў выдаткаў на асваенне (10\%).
\end{Explanation}

Апроч таго, арганізацыя-распрацоўшчык нясе выдаткі на суправаджэнне праграмнага забеспячэння, якое вызначаецца па формуле (\ref{eqn: eqn18}):

\begin{equation}
    \label{eqn: eqn18}
    \text{P}_\text{c} = \frac{\text{C}_\text{П} \cdot \text{H}_\text{c}}{100}
\end{equation}
\begin{Explanation}
    \item[дзе] $\text{P}_\text{с}$ -- выдаткі на суправаджэнне;
    \item $\text{H}_\text{с}$ -- нарматыў выдаткаў на суправаджэнне (20\%).
\end{Explanation}

Агульная сума выдаткаў на распрацоўку ($\text{C}_\text{ПП}$) з выдаткамі на асваенне і суправаджэнне, як поўны сабекошт праграмнага прадукту, вызначаецца па формуле (\ref{eqn: eqn19}):

\begin{equation}
    \label{eqn: eqn19}
    \text{C}_\text{ПП} = \text{C}_\text{П} + \text{Р}_\text{а} + \text{Р}_\text{с}
\end{equation}

Вынікі разліку пералічаных вышэй паказальнікаў прадстаўлены ў табліцы \ref{table: last}.

\clearpage

\begin{table}[htp]
    \caption{Разлік сабекошту і адпускной цаны праграмнага прадукту}
    \begin{tabularx}{\textwidth}{ 
        | >{\centering\arraybackslash}X 
        | >{\centering\arraybackslash}X 
        | >{\centering\arraybackslash}X |
    }
    \hline
    Назва артыкулу выдаткаў & Умоўныя адзнакі & Значэнне, руб \\
    \hline
    Адлічэнні ў фонд сацыяльнай абароны насельніцтва & ФСАН & 37,07 \\
    \hline
    Адлічэнні ў Белдзяржстрах & БДС & 0,65 \\
    \hline
    Матэрыялы і складнікі & М & 2,97 \\
    \hline
    Навуковыя камандыроўкі & $\text{Р}_\text{НК} $ & 9,91 \\
    \hline
    Іншыя правыя выдаткі & ІВ & 19,82 \\
    \hline
    Накладныя выдаткі & $\text{Р}_\text{Н}$ & 9,91 \\
    \hline
    Поўны сабекошт & $\text{С}_\text{П}$ &  169,54 \\
    \hline
    Планавы прыбытак & П & 25,43 \\
    \hline
    Падатак на дадатковы кошт & ПДК & 33,90 \\
    \hline
    Адпускная цана & $\text{Ц}_\text{АДП}$ & 228,87 \\
    \hline
    \end{tabularx}
    \label{table: last}
\end{table}

\vspace{-\baselineskip}

\subsection{Вынікі}

Падас выканання разліку сабекошту і цаны праграмнага прадукту атрыманы наступныя значэнні: асноўная сумарная заработная плата з улікам прэміі склала 109,03 рублі, адлічэнні ў фонд сацыяльнай абароны склалі 37,07 рублёў, адлічэнні ў Белдзяржстрах на страхаванне ад няшчасных выпадкаў склалі 0,65 рублёў.

Адпускная цана праграмнага прадукту, з улікам выдаткаў на аплату працы, адлічэнняў на сацыяльныя патрэбы, выдаткі на матэрыялы, выдаткі на навуковыя камандзіроўкі, іншыя прамыя выдаткі, падатку на дадатковы кошт і чаканага прыбытку, склада 228,87 рублёў, пры гэтым планавы прыбытак склаў 25,43 рублі.

Аднак, неабходна заўважыць, што нягледзячы на невялікі прамы прыбытак праграмнага прадукту, ён дазволіць арганізацыі значна скараціць выдаткі на:
\begin{enumerate}
    \item сеткавае абсталяванне, так як выкарыстанне дадзенага праграмнага прадукту дазваляе выкарыстоўваць больш простае сеткавае абсталяванне без уплыву на якасць прадастаўленых паслуг;
    \item заработную плату супрацоўнікаў, так як дадзены праграмны прадукт спрашчае канфігурацыю сеткавага абсталявання і яе падтрымку, што дазваляе аптымізаваць работу сеткавых інжынераў, што, у сваю чаргу, прывядзе да змяншэння штату сеткавых інжынераў.
\end{enumerate}
