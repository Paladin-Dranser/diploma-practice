\section{Тэхнічна-эканамічнае абгрунтаванне}

\subsection{Кароткая характарыстыка распрацаванага праграмнага забеспячэння}

Дыпломны праект прадстаўляе сабой праграмнае мадэліраванне гібрыднай SDN сеткі на базе EIGRP пратакола. У сетку ўводзіцца дадатковы элемент-кантролер, які ажыццяўляе маніторынг характарыстык сеткавых
элементаў і прыняцця рашэння пра пераразлік метрыкі і перабудовы сеткавай тапалогіі на маршрутызатары.

Асноўныя перавагі пабудовы гібрыднай SDN мадэлі:
\begin{enumerate}
    \item працэсы перадачы і кіраванне данымі падзелены. Гэта значыць, што маршрутызатары
        адказваюць толькі за функцыю перадачы даных, у той час як за логіку маршрутызацыі
        адказвае адмысловы кантролер;
    \item кіраванне сеткай цэнтралізавана пры дапамозе ўніфіцыраваных праграмных сродкаў.
        Так як кіраванне сеткай ажыццяўляецца пры дапамозе знешніх праграмных сродкаў, гэта
        дазваляе лёгка ўносіць неабходныя настройкі на ўсе ўстройства ў сетцы.
        Пры гэтым неабходна дадаць, што дадзены прынцып дазваляе наблізіць канфігурацыю
        сеткавых устройств да канцэпцыі <<інфраструктура як код>>, што забяспечвае
        аўтаматызацыю канфігурацыі сеткі;
    \item эканомія на сеткавых устройствах. Так як кіруючая функцыя рэалізаваная на асобным
        кантролеры, то сеткавае ўстройства больш не патрабуе наяўнасці вялікіх рэсурсаў
        (працэсар, памяць і да т. п.), што спрашчае яго структуру, а значыць і кошт устройства;
    \item памяншэнне працазатрат на суправаджэнне сеткі. Так як ўся інфармацыя і канфігурацыя
        маршрутызатараў знаходзіцца ў адным месцы, то супрацоўнікі патрабуецца менш часу для
        карэктыроўкі канфігурацыі пры неабходнасці;
    \item павелічэнне эфектыўнасці выкарыстання рэсурсаў секті. Кантролер дазваляе адсочваць
        характарыстыкі ўстройств у рэльным часе, што дазваляе аўтаматычна змяняць канфігурацыю
        ўстройства для дасягнення аптымальнай канфігурацыі ў кожны момант часу.
\end{enumerate}

\subsection{Разлік сабекошту і цаны праграмнага прадукту}

Першым крокам неабходна вызначыць вартасную ацэнку распрацаванага праграмнага прадукту. Яна прадугледжвае складанне каштарысу выдаткаў, на падставе якога вызначаецца сабекошт і адпускная цана распрацаванага праграмнага прадукту.
Каштарыс выдаткаў ўключае наступныя выдаткі:
\begin{enumerate}
    \item выдаткі на аплату працы выканаўцаў;
    \item адлічэнні на сацыяльныя патрэбы;
    \item матэрыялы і камплектуючыя;
    \item выдаткі на навуковыя камандзіроўкі;
    \item іншыя прамыя выдаткі;
    \item накладныя выдаткі.
\end{enumerate}

У артыкул «Выдаткі на аплату працы» ўключаюцца асноўная і дадатковая заработная плата ўсіх супрацоўнікаў, занятых распрацоўкай праграмнага забеспячэння для SDN кантролера. У дадзеным выпадку напісаннем праграмы займаецца адзін інжынер-праграміст.

Разлік асноўнай заработнай платы выконваецца па формуле (\ref{eqn: eqn1}):
\begin{equation}
    \label{eqn: eqn1}
    \text{З}_\text{o} = \sum_{i=1}^{n} \text{T}_\text{di} \cdot \text{Ф}_\text{П} \cdot \text{K}_\text{ПР}
\end{equation}
\begin{Explanation}
    \item[дзе] $\text{T}_\text{di}$ --- дзённая тарыфная стаўка $i$-га выканаўца, руб;
    \item $\text{Ф}_\text{п}$ --- планавы фонд працоўнага часу - час, у перыяд якога
        супрацоўнік $i$-ай катэгорыі прымаў удзел у распрацоўцы, дзён;
    \item $\text{К}_\text{ПР}$ --- каэфіцыент прэміравання за выкананне планавых
        паказальнікаў работы, роўны 1,4.
\end{Explanation}

Дзённая тарыфная стаўка $i$-га выканаўца разлічваецца наступным
чынам:
\begin{equation}
    \text{T}_\text{дi} = \text{T}_\text{мі} / \text{Ф}_\text{р}
\end{equation}
\begin{Explanation}
    \item[дзе] $\text{T}_\text{мі}$ --- месячная тарыфная стаўка $i$-га выканаўца, руб;
    \item $\text{Ф}_\text{р}$ --- сярэднемесячны фонд працоўна часу, дні.
\end{Explanation}

Разліковая норма працоўнага часу ў днях на 2020 год пры пяцідзённым працоўным тыдні складае 256 дзён. Тады сярэдне месячны фонд працоўнага часу складае прыблізна 21 дзень.

Месячная тарыфная стаўка кожнага выканаўца вызначаецца па
формуле (\ref{enq: enq3}):
\begin{equation}
    \label{enq: enq3}
    \text{T}_\text{M} = \text{T}_\text{М1p} \cdot \text{T}_\text{K}
\end{equation}
\begin{Explanation}
    \item[дзе] $\text{T}_\text{M1p}$ --- дзеючая месячная тарыфная стаўка першага разраду (41 рубель);
    \item $\text{T}_\text{K}$ --- тарыфны каэфіцыент (1,21).
\end{Explanation}

Дадатковая заработная плата персанала ўключае выплаты, якія
прадугледжаны дзеючым заканадаўствам за непрапрацаваны час ---
водпуск, выплата за выслугу гадоў і да таго падобнае.
Дадатковая заработная плата вызначаецца па формуле (\ref{eqn: eqn4}):

\begin{equation}
    \label{enq: enq4}
    \text{З}_\text{Д} = \text{З}_\text{o} \cdot \text{H}_\text{Д} / 100\%
\end{equation}
\begin{Explanation}
    \item[дзе] $\text{H}_\text{Д}$ --- нарматыў дадатковай заработнай платы (10\%).
\end{Explanation}

Агульныя затраты на аплату працы (ЗАП) вызначаецца па формуле
(\ref{eqn: eqn5}):
\begin{equation}
    \label{eqn: eqn5}
    \text{ЗАП} = \text{З}_\text{o} + \text{З}_\text{Д}
\end{equation}

Разлік асноўнай заработнай платы персаналу прадстаўлены
ў \ref{table: 5.1}
