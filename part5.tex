\section{Тэхнічна-эканамічнае абгрунтаванне}

\subsection{Кароткая характарыстыка распрацаванага праграмнага забеспячэння}

Дыпломны праект прадстаўляе сабой праграмнае мадэліраванне гібрыднай SDN сеткі на базе EIGRP пратакола. У сетку ўводзіцца дадатковы элемент-кантролер, які ажыццяўляе маніторынг характарыстык сеткавых
элементаў і прыняцця рашэння пра пераразлік метрыкі і перабудовы сеткавай тапалогіі на маршрутызатары.

Асноўныя перавагі пабудовы гібрыднай SDN мадэлі:
\begin{enumerate}
    \item працэсы перадачы і кіраванне данымі падзелены. Гэта значыць, што маршрутызатары
        адказваюць толькі за функцыю перадачы даных, у той час як за логіку маршрутызацыі
        адказвае адмысловы кантролер;
    \item кіраванне сеткай цэнтралізавана пры дапамозе ўніфіцыраваных праграмных сродкаў.
        Так як кіраванне сеткай ажыццяўляецца пры дапамозе знешніх праграмных сродкаў, гэта
        дазваляе лёгка ўносіць неабходныя настройкі на ўсе ўстройства ў сетцы.
        Пры гэтым неабходна дадаць, што дадзены прынцып дазваляе наблізіць канфігурацыю
        сеткавых устройств да канцэпцыі <<інфраструктура як код>>, што забяспечвае
        аўтаматызацыю канфігурацыі сеткі;
    \item эканомія на сеткавых устройствах. Так як кіруючая функцыя рэалізаваная на асобным
        кантролеры, то сеткавае ўстройства больш не патрабуе наяўнасці вялікіх рэсурсаў
        (працэсар, памяць і да т. п.), што спрашчае яго структуру, а значыць і кошт устройства;
    \item памяншэнне працазатрат на суправаджэнне сеткі. Так як усё інфармацыя і канфігурацыя
        маршрутызатараў знаходзіцца ў адным месцы, то супрацоўнікі патрабуецца менш часу для
        карэктыроўкі канфігурацыі пры неабходнасці;
    \item павелічэнне эфектыўнасці выкарыстання рэсурсаў секті. Кантролер дазваляе адсочваць
        характарыстыкі ўстройств у рэльным часе, што дазваляе аўтаматычна змяняць канфігурацыю
        ўстройства для дасягнення аптымальнай канфігурацыі ў кожны момант часу.
\end{enumerate}


