\section{Мадэліраванне прататыпа размеркавання патокаў трафіка ў сетках SDN}

У дадзеным раздзеле разглядаецца мадэль невялікай гібрыднай SDN сеткі, якая працуе
пад кантролем вышэапісаннага алгарытму разліку нагрузкі.

Для сімуляцыі сеткі выкарыстоўваецца GNS3 праграма.
GNS3 --- гэта графічны сімулятар сеткі, які дазваляе змадэліраваць віртуальную
сетку з маршрутызатараў і віртуальных машын.

\subsection{Мадэль віртуальнай сеткі}

На малюнку \ref{img: model} прадстаўлена мадэль віртуальнай сеткі з 3 маршрутызатараў і 1 кантролера.

\begin{figure}[h!]
    \centering
    \includegraphics[width=\textwidth]{model.png}
    \caption{Мадэль віртуальнай сеткі}
    \label{img: model} 
\end{figure}

З малюнка \ref{img: model} бачым, што маршрутызатар R1 падключаны да сеткі 10.10.22.0/24 і сеткі 192.168.254.0/24;
маршрутызатар R2 падключаны да сетак 10.10.22.0/24, 10.10.11.0/24 і сеткі 10.10.33.0/24 (прадстаўленая ў відзе loopback інтэрфейса);
маршрутызатар R3 падключаны да 10.10.11.0/24 і 192.168.254.0/24.

Знешні кантролер падключаны да кожнага маршрутызатара. Неабходна заўважыць, што кантролер можа падключацца да маршрутызатараў
як напрамую (як праўдстаўлена на малюнку \ref{img: model}), так і праз іншыя маршрутызатары.

У якасці маршрутызатараў выкарыстоўваюцца маршрутызатары мадэлі CSR1000v (Cisco Cloud Services Router 1000V Series)
з наступнымі параметрамі:
\begin{enumerate}
    \item пратакол маршрутызацыі EIGRP;
    \item прапускная здольнасць на інтэрфейсу ў памеры 10 Кбіт/с;
    \item каэфіцыенты EIGRP метрыкі: $K1 = 1$, $K2 = 1$, $K3 = 1$, $K4 = 0$, $K5 = 0$.
\end{enumerate}

У якасці кантролера выкарыстоўваецца камп'ютар з аперацыйнай сістэмай Debian, на якім
запушчана праграма маніторынгу і пераразліку метрыкі.

\subsection{Першапачатковыя параметры маршрутызатараў}

У дадзеным падраздзеле паказаны пачатковыя параметры на маршрутызатарах (інтэрфейсы, табліца EIGRP маршрутаў, EIGRP тапалогія)
перад падачай трафіка ў сетку.

\subsubsection{Маршрутызатар R1.}

У лістынгу \ref{lst: R1 before/interfaces} прадстаўлена кароткае апісанне ўключаных інтэрфейсаў на маршрутызатары R1.

\lstinputlisting[caption={Уключаныя інтэрфейсы на маршрутызатары R1},%
                            label={lst: R1 before/interfaces},%
                            language=bash]{R1/before/interfaces.txt}

У лістынгу \ref{lst: R1 before/routes} прадстаўлены ўсе даступныя EIGRP маршруты на маршрутызатары R1.


\lstinputlisting[caption={EIGRP маршруты на маршрутызатары R1},%
                            label={lst: R1 before/routes},%
                            language=bash]{R1/before/routes.txt}

З лістынга \ref{lst: R1 before/routes} бачым, што маршрутызатар R1
вывучыў 2 EIGRP маршруты ад суседзяў.
Так для сеткі 10.10.11.0/24 маршрутызатар R1 мае 2 маршруты (балансіроўка
маршрутаў, так як іх метрыкі супадаюць)
праз маршрутызатар R2 (10.10.22.22) і маршртузатар R3 (via 192.168.254.2),
а для сеткі 10.10.33.0/24 мае 1 маршрут праз маршрутызатар R2 (via 10.10.22.22).

У лістынгу \ref{lst: R1 before/topology} прадстаўлена EIGRP тапалогія сеткі на
маршрутызатары R1.

\lstinputlisting[caption={EIGRP тапалогія сеткі на маршрутызатары R1},%
                            label={lst: R1 before/topology},%
                            language=bash]{R1/before/topology.txt}

У EIGRP тапалогіі сеткі можна ўбачыць для кожнай EIGRP сеткі бягучага
Successor-а (маршрутызатар з меншым значэннем FD), а таксама запаснога
Successor-а пры яго наяўнасці.

Напрыклад, для маршрутызатара R1 для сеткі 10.10.11.0/24 існуе 2 Successor-а,
што стала магчымым для ўзнікненна балансіроўкі для дадзенага маршруту.
У той жа час для 10.10.33.0/24 даступны толькі 1 Successor, хаця дадзеная
сетка можа быць дасягнута праз 2 маршрутызатары (via 10.10.22.22 і via 192.168.254.2).

