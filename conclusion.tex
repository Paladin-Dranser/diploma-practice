\sectionWithoutNumber{Заключэнне}

У дадзенай дыпломнай рабоце выконвалася праграмнае мадэліраванне
патокаў трафіка ў сетках SDN.

Для дасягнення пастаўленай мэты
была разгледжана канцэпцыя праграмна-вызначаных сетак, асноўныя прынцыпы
яе пабудовы. Таксама быў праведзены аналіз пратаколаў маршрутызацыі,
якія прымяняюцца ў традыцыйных камп'ютарных сетках, было праведзена
параўнанне іх прыдатнасці ды выкарыстання ў праграмна-вызначаных сетках
гібрыднай мадэлі. На базе пратакола маршрутызацыі EIGRP быў распрацаваны
алгарытм пераразліку тапалогіі сеткі ў залежнасці ад паказальніка нагрузкі
на інтэрфейсах маршрутызатара. У графічным сеткавым сімулятары GNS3 была
прадэманстравана праца распрацаванага алгарытму ў камп'ютарнай сеткі з
3 маршрутызатараў і 1 кантролера.

Распрацаваны алгарытм пераразліку тапалогіі сеткі дазваляе адаптаваць
тапалогію сеткі ў рэальным часе ў залежнасці ад нагрузкі на інтэрфейсах
сеткавых устройств, што ў сваю чаргу дазваляе не толькі аптымізаваць
выкарыстанне рэсурсаў сеткавымі ўстройствамі, аданак і палепшыць якасць паслуг,
якія прадастаўляюцца кліентам, нават у час найбольшай нагрузкі.

У планах далейшага развіцця праграмнага забеспячэння алгарытму
пераразліку тапалогіі сеткі наступныя крокі:
\begin{enumerate}
    \item вынесці ўзровень захавання даных у асобны кампанент (захаванне нагрузкі на інтэрфейсах ажыццяўляць у базе даных), што дазволіць
    пабудаваць дэцэнтралізавуную інфраструктуру кантролера для павелічэння
    адмоваўстойлівасці і прадукцыйнасці кантролера;
    \item рэалізаваць падтрымку атрымання неабходнай інфармацыі з
    сеткавага ўстройства любога вытворца (напрыклад, Cisco, Huawei, ZTE).
\end{enumerate}

У тэхнічна-эканамічным раздзеле на аснове сабекошту распрацаванага прадукту
была падцверджана мэтазгоднасць распрацоўкі дадзенага праграмнага забеспячэння.

З усяго вышэйапісанага можам зрабіць вынік, што мэта дыпломнай працы была
дасягнута ў поўным аб'ёме (распрацоўка алгарытму пераразліку тапалогіі
сеткі і праграмнае мадэліраванне яго працы). Атрыманы прататып
праграмнага забеспячэння выконвае ўсе неабходныя задачы і яго распрацоўка
і далейшае развіццё з'яўляецца эканамічна мэтазгодным.
