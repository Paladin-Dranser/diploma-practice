% Метод управления трафиком в гибридной программно-определяемой сети
\section{Аналіз пратаколаў маршрутызацыі і распрацоўка праграмнага забеспячэння для кіравання трафікам ў гібрыдных сетках SDN}

\subsection{Тыпы алгарытмаў маршрутызацыі}

Алгарытмы маршрутызацыі могуць быць класіфікаваны па тыпах.
Напрыклад, алгарытмы могуць быць:
\begin{enumerate}
    \item статычнымі альбо дынамічнымі;
    \item аднамаршрутнымі альбо шматмаршрутнымі;
    \item аднаўзроўневымі альбо іерархічнымі;
    \item з інтэлектам у галоўнай вылічальнай машыне альбо маршрутызатары;
    \item унутрыдаменнымі і міждаменнымі;
    \item алгарытмамі стану канала альбо дыстанцыйна-вектарнымі.
\end{enumerate}

Статычныя алгарытмы маршрутызацыі наогул наўрад ці з'яўляюцца
алгарытмамі.
Размеркаванне статычных табліц маршрутызацыі
устанаўліваецца адміністратарам сеткі да пачатку маршрутызацыі. Яно не
змяняецца, калі толькі адміністратар сеткі не зменіць яго. Алгарытмы,
якія выкарыстоўваюць статычныя маршруты, простыя для распрацоўкі і добра
працуюць у асяроддзях, дзе трафік сеткі адносна прадказальны, а схема
сеткі адносна простая.

Так як статычныя сістэмы маршрутызацыі не могуць рэагаваць на
змены ў сетцы, яны, як правіла, лічацца непрыдатнымі для сучасных
буйных, пастаянна зменлівых сетак. Большасць дамінуючых
алгарытмаў маршрутызацыі --- дынамічныя.

Дынамічныя алгарытмы маршрутызацыі падладжваюцца да зменлівых акалічнасцяў сеткі ў маштабе рэальнага часу. Яны выконваюць гэта шляхам аналізу паведамленняў аб абнаўленні маршрутызацыі. Калі ў паведамленні паказваецца, што мела месца змяненне сеткі, праграмы маршрутызацыі пералічваюць маршруты і рассылаюць новыя паведамленні аб карэкціроўцы маршрутызацыі. Такія паведамленні праходзяць па ўсёй сетцы, стымулюючы маршрутызатары зноўку выконваць свае алгарытмы і адпаведным чынам змяняць табліцы маршрутызацыі.

Дынамічныя алгарытмы маршрутызацыі могуць дапаўняць статычныя маршруты там, дзе гэта неабходна. Напрыклад, можна распрацаваць <<маршрут апошняй надзеі>> (г.зн. маршрут, у які адсылаюцца ўсе неадпраўленыя па вызначаным маршруце пакеты). Такі маршрут выконвае ролю сховішча неадпраўленых 
пакетаў, гарантуючы, што ўсе паведамленні будуць хаця б пэўным чынам апрацаваны.

Некаторыя складаныя пратаколы маршрутызацыі забяспечваюць мноства маршрутаў да аднаго і таго ж пункта прызначэння. Такія шматмаршрутные алгарытмы робяць магчымай мультыплексную перадачу трафіка па шматлікіх лініях; аднамаршрутныя алгарытмы не могуць рабіць гэтага.
Перавагі шматмаршрутных алгарытмаў відавочныя --- яны могуць забяспечыць значна большую прапускную здольнасць і надзейнасць.

Некаторыя алгарытмы маршрутызацыі аперуюць ў плоскай
прасторы, у той час як іншыя выкарыстоўваюць іерархічную маршрутызацыю. У аднаўзроўневай сістэме маршрутызацыі ўсе маршрутызатары роўныя ў адносінах адзін да аднаго. У іерархічнай сістэме маршрутызацыі некаторыя маршрутызатары фарміруюць тое, што складае аснову маршрутызацыі.
Пакеты з небазовых маршрутызтараў перамяшчаюцца да базавых і прапускаюцца праз іх да таго часу, пакуль не дасягнуты агульнай вобласці пункта прызначэння. Пачынаючы з гэтага моманту, яны перамяшчаюцца ад апошняга базавага маршрутызатара праз адзін або некалькі небазовых маршрутызатараў да канчатковага пункта прызначэння.

Сістэмы маршрутызацыі часта ўсталёўваюць лагічныя групы вузлоў, даменамі, або аўтаномнымі сістэмамі (AS), або абласцямі. У іерархічных сістэмах адны маршрутызатары якога-небудзь дамена могуць звязвацца з маршрутызатарамі іншых даменаў, у той час як іншыя маршрутызатары гэтага дамена могуць падтрымліваць сувязь з маршрутызатарамі толькі ў межах свайго дамена. У вельмі буйных сетках могуць існаваць дадатковыя іерархічныя ўзроўні. Маршрутызатары найвышэйшага іерархічнага ўзроўню ўтвараюць базу маршрутызацыі.

Асноўнай перавагай іерархічнай маршрутызацыі з'яўляецца тое, што яна імітуе арганізацыю большасці кампаній і такім чынам, вельмі добра падтрымлівае іх схемы передачы трафіка. Большая частка сеткавай сувязі мае месца ў межах груп невялікіх кампаній (даменаў). Унутрадаменны маршрутызатарау неабходна ведаць толькі пра іншыя маршрутызатары ў межах свайго дамена, таму іх алгарытмы маршрутызацыі могуць быць спрошчанымі. Адпаведна можа быць паменшаны і трафік абнаўлення маршрутызацыі, які залежыць ад выкарыстоўванага алгарытму маршрутызацыі.

Некаторыя алгарытмы маршрутызацыі мяркуюць, што канчатковы вузел крыніцы вызначае ўвесь маршрут. Звычайна гэта называюць маршрутызацыяй ад крыніцы. У сістэмах маршрутызацыі ад крыніцы маршрутызатары дзейнічаюць проста як устойства захоўвання і перасылкі пакета, без ўсякіх вылічэнняў адсылаючы яго да наступнага маршрутызатара.

Іншыя алгарытмы мяркуюць, што галоўныя вылічальныя машыны нічога не ведаюць пра маршруты. Пры выкарыстанні гэтых алгарытмаў маршрутызатары вызначаюць маршрут праз аб'яднаную сетку, грунтуючыся на сваіх уласных разліках.

У першай сістэме, разгледжанай вышэй, інтэлект маршрутызацыі знаходзіцца ў галоўнай вылічальнай машыне. У сістэме, разгледжанай у другім выпадку, інтэлектам маршрутызацыі надзелены самі маршрутызатары.

Кампраміс паміж маршрутызацыяй з інтэлектам у галоўнай
вылічальнай машыне і маршрутызацыяй з інтэлектам ў маршрутызатары дасягаецца шляхам супастаўлення аптымальнасці маршруту з непрадукцыйнымі аб'ёмамі трафіка. Сістэмы з інтэлектам у галоўнай вылічальнай машыне часцей выбіраюць найлепшыя маршруты, так як яны, як правіла, знаходзяць усе магчымыя маршруты да пункта прызначэння, перш чым пакет будзе сапраўды дасланы. Затым яны выбіраюць найлепшы маршрут, грунтуючыся на вызначэнні аптымальнасці дадзенай канкрэтнай сістэмы. Аднак акт вызначэння ўсіх маршрутаў часта патрабуе значнага аб'ёму трафіка і вялікай колькасці часу.

Некаторыя алгарытмы маршрутызацыі дзейнічаюць толькі ў межах даменаў; іншыя --- як у межах даменаў, так і паміж імі. Прырода гэтых двух тыпаў алгарытмаў розная. Таму зразумела, што аптымальны алгарытм унутрыдаменнай маршрутызацыі не абавязкова будзе аптымальным
алгарытмам міждаменнай маршрутызацыі.

Алгарытмы стану канала накіроўваюць патокі маршрутнай інфармацыі ва ўсе вузлы аб'яднанай сеткі. Аднак кожны маршрутызатара пасылае толькі тую частку маршрутнай табліцы, якая апісвае стан яго ўласных каналаў. Алгарытмы дыстанцыйна-вектарныя патрабуюць ад каждогo маршрутызатара пасылкі ўсёй або часткі сваёй маршрутнай табліцы, але толькі сваім суседзям.

Алгарытмы стану каналаў фактычна накіроўваюць невялікія карэкціроўкі па ўсіх напрамках, у той час як дыстанцыйна-вектарныя алгарытмы адсылаюць буйнейшыя карэкціроўкі толькі ў суседнія маршрутызатары.
Адрозніваючыся хутчэйшай збежнасцю, алгарытмы стану каналаў некалькі менш схільныя да стварэння цыклаў маршрутызацыі, чым дыстанцыйна-вектарныя алгарытмы. З іншага боку, алгарытмы стану канала характарызуюцца больш складанымі разлікамі, патрабуючы большай працэсарнай магутнасці і памяці, чым дыстанцыйна-вектарныя алгарытмы. З прычыны гэтага, рэалізацыя і падтрымка алгарытмаў стану канала можа быць больш дарагой. Нягледзячы на іх адрозненні, абодва тыпы алгарытмаў добра функцыянуюць пры самых розных абставінах.

У алгарытмах маршрутызацыі выкарыстоўваецца шмат розных паказчыкаў. Складаныя алгарытмы маршрутызацыі пры выбары маршруту могуць грунтавацца на мностве паказчыкаў, камбінуючы іх такім чынам, што ў выніку атрымліваецца адзін асобны (гібрыдны) паказчык. Ніжэй прыведзены некаторыя паказчыкі, якія могуць выкарыстоўваюцца ў алгарытмах маршрутызацыі:
\begin{enumerate}
    \item даўжыня маршруту;
    \item надзейнасць;
    \item затрымка;
    \item шырыня паласа прпускания;
    \item нагрузка;
    \item кошт сувязі.
\end{enumerate}

Даўжыня маршруту з'яўляецца найбольш агульным паказчыкам маршрутызацыі. Некаторыя пратаколы маршрутызацыі дазваляюць адміністратарам сеткі прызначаць адвольныя кошты на кожны канал сеткі. У гэтым выпадку даўжынёй маршруту з'яўляецца сума выдаткаў, звязаных з кожным каналам, які быў пройдзены. Іншыя пратаколы маршрутызацыі вызначаюць <<колькасць перасыланняў>>, гэта значыць паказчык, які характарызуе колькасць праходаў, якія пакет павінен здзейсніць на шляху ад крыніцы да пункту прызначэння.

Надзейнасць, у кантэксце алгарытмаў маршрутызацыі, адносіцца да надзейнасці кожнага канала сеткі (звычайна апісваецца у тэрмінах суадносін біт/памылка). Некаторыя каналы сеткі могуць адмаўляць часцей, чым іншыя. Адмовы адных каналаў сеткі могуць быць ліквідаваны лягчэй або хутчэй, чым адмовы іншых каналаў. Пры прызначэнні адзнак надзейнасці могуць быць прыняты ў разлік любыя фактары надзейнасці. Ацэнка надзейнасці звычайна прызначаюцца каналам сеткі адміністратарамі сеткі. Як правіла, гэта адвольныя лічбавыя велічыні.

Пад затрымкай маршрутызацыі звычайна разумеюць адрэзак часу, неабходны для руху пакета ад крыніцы да пункта прызначэння праз аб'яднаную сетку. Затрымка залежыць ад шматлікіх фактараў, уключаючы паласу прапускання прамежкавых каналаў сеткі, чэргі ў порт кожнага
маршрутызатара на шляху перадачы пакета, перагружанасць сеткі на ўсіх прамежкавых каналах сеткі і фізічная адлегласць, на якую неабходна перамясціць пакет. Так як тут мае месца ўплыў некалькіх важных зменных, затрымка з'яўляецца найбольш агульным і карысным паказчыкам.

Паласа прапускання прыраўноўваецца да наяўнай магутнасці трафіку якога-небудзь канала. Пры іншых роўных паказчыках, канал Ethernet 10 Мбіт пераважны любой арэндаванай лініі з паласой прапускання 64 Кбіт/сек. Хоць паласа прапускання з'яўляецца ацэнкай максімальна дасягальнай прапускной здольнасці канала, маршруты, якія праходзяць праз каналы з большай паласой прапускання, не абавязкова будуць лепшымі за маршрутам, якія праходзяць праз менш хуткадзейныя каналы.

\subsection{Аналіз пратаколаў маршрутызацыі ў канцэпцыі SDN сетак}

Як згадвалася ў першым раздзеле дадзенай работы традыцыйныя пратаколы
маршрутызацыі, такія як OSPF, RIP, BGP, EIGRP і іншыя, не былі прызначаныя
для выкарыстання ў праграмна вызаначаных сетках.

У дадзеным падраздзеле правядзем кароткі аналіз асноўных пратаколаў маршрутызацыі на магчымасць іх выкарыстоўвання  ў
праграмна-вызначаных сетак.

Дыстанцыйна-вектарны пратакол маршрутызацыі RIP --- адзін з самых простых пратаколаў маршрутызацыі, які можа выкарыстоўвацца ў невялікіх камп'ютарных сетках.
У якасці метрыкі пратакол маршрутызацыі RIP выкарыстоўвае колькасць неабходных пераходаў паміж маршрутызатарамі для дасягнення пункту
прызначэння. Колькасць пераходаў абмежаваная 15 маршрытазатарамі.
Прастата пратакола маршрутызацыі RIP ў вызначэнні метрыкі і абмежаванне
колькасці транзітных маршрутызатраўа
з'яўляюцца галоўным фактарамі немэтазгоднасці выкарыстоўвання дадзенага
пратакола маршрутызацыі ў SDN сетках, якія прадугледжваюць гібкую канфігурацыю вызначэння аптымальнага маршруту ў сетцы.

Пратаколы маршрутызацыі стану канала, такія як IS-IS і OSPF, з'яўляюцца
найбольш распаўсюджанымі ў традыцыйных сетках. Пратаколы маршрутызацыі
стану канала прадастаўляюць багатыя магчымасці для канфігурацыі сеткі,
хуткую збежнасць, падтрымку іерархіі сеткі, якая гарантуе адсутнасць
петляў у сетцы.

Апошняя ўласцівасць дасягаецца за кошт пабудовы усімі маршрутызатарамі сеткі адзінай базы даных сеткавых падключэнняў --- LSDB (Link State DataBase). Фарміраванне LSDB заснавана, у сваю чаргу, на фарміраванні суседскіх адносін паміж маршрутызатарамі. Фарміраванне і падтрыманне суседскіх адносін неабходна для магчымасці пабудовы поўнай тапалогіі сеткі (фарміравання LSDB). Фарміраванне суседскіх адносін адбываецца па-рознаму, у залежнасці ад тыпу сеткі, абодва пратаколы (OSPF, IS-IS) вылучаюць два асноўных тыпу сетак: шырокавяшчальныя (Ethernet), NBMA (Nonbroadcast
Multiple Access, такія варыяцыі як пункт-пункт, пункт-мультыпункт).
Бягучыя спецыфікацыі пратаколу OpenFlow, якія выкарыстоўваецца для пабудовы класічнай мадэлі SDN сетак, вызначаюць злучэння выключна як злучэння Ethernet, гэта значыць, што  з пункту гледжання пратаколаў класа link-state гэта шырокавяшчальныя сеткі. Для такога тыпу сеткі суседскія адносіны фарміруюцца паміж усімі маршрутызатарамі, падлучанымі да сеткі.

Усе вылічэнні пратаколаў і пабудовы табліц, у выпадку класічнай мадэлі SDN, вырабляюцца на OpenFlow-кантролеры. У гэтым выпадку, улічваючы, што OpenFlow-камутатары не могуць фармаваць паміж сабой суседскія адносіны, LSDB можа фармавацца на кантролеры, зыходзячы толькі з двух параметраў:
\begin{enumerate}
    \item падключаных непасрэдна да маршрутызатараў сеткі;
    \item Ethernet (шырокавяшчальнага тыпу) пакдлючэнняў.
\end{enumerate}

Таму у тым выпадку, калі ў сетцы выкарыстоўваюцца падключэнні тыпу NBMA, на Оpenflow-кантролеры будзе немагчыма, у аўтаматычным рэжыме,
карэктна аднавіць поўную тапалогію сеткі (LSDB), што пацягне за сабой некарэктную маршрутызацыю.
Напрыклад, у выпадку падключэння маршрутызатараў адзін да аднаго праз сетку тыпу NBMA з выкарыстаннем тапалогіі пункт-мультыпункт, Оpenflow-кантролер адновіць дадзеную тапалогію як падлучэнне «кожны з кожным» праз шырокавяшчальную сетку праз названыя вышэй уласцівасці.

Акрамя таго, паказанай вышэй праблемы, вынікае цэлы шэраг іншых праблем, напрыклад:
\begin{enumerate}
    \item некарэктная маршрутызацыя BGP пратакола. Выклікана гэта тым, што ў сваёй рабоце BGP разлічвае на работу пратаколаў IGP (OSPF, IS-IS). Адпаведна, некарэктная маршрутызацыя пратаколаў IGP выкліча некарэктную маршрутызацыю пратакола BGP;
    \item немагчымасць выкарыстоўваць пратаколы маршрутызацыі multicast трафіку, якія з'яўляюцца дэ-факта стандартам у галінe тэлекамунікацый (PIM-DM, PIM-SM). Справа ў тым, што дадзеныя пратаколы выкарыстоўваюць для RPF-праверак табліцу маршрутызацыі, пабудаваную звычайнымі пратаколамі
    маршрутызацыі (unicast-трафіка). Ізноў некарэктная работа пратаколаў
    маршрутызацыі прывядзе да некарэктнай работы пратаколаў PIM-DM, PIM-SM.
\end{enumerate}

Пратакол EIGRP з'яўляецца ўдасканаленым дыстанцыйна-вектарным
пратаколам маршрутызацыі. Хаця пратакол EIGRP таксама патрабуе ўстанаўлення
суседства з іншымы маршрутызатарымі, падобна пратаколам OSPF і IS-IS, аднак
суседства ў пратаколе EIGRP мае іншыя прыроду. Маршрутызатары з'яўляюцца суседзямі, калі яны напрамую падключаныя адзін да аднаго. Дадзеная асаблівасць суседства дазваляе пабудаваць тапалогіі сеткі на кантролеры.
Апроч таго пратакол маршрутызацыі EIGRP можа выкарыстоўваць да 5 відаў метрык, пры гэтым па ўмаўчанні выкарыстоўваецца толькі 2 віды:
прапускная здольнасць і затрымка.
Разнастайнасць відаў метрык дазваляе распрацоўваць дадатковыя ўмовы,
якія будуць уплываць на выбар аптымальнага маршруту да пункта прызначэння,
што з'яўляецца значнай перавагай дадзенага пратакола маршрутызацыі пры
пабудове гібрыднай мадэлі SDN сетак.

З вышэй апісанага можам зрабіць вынік, што для пабудовы гібрыднай мадэлі
SDN сетак аптымальна выкарыстоўваць EIGRP пратакол, так як ён мае багатыя
дадатковыя магчымасці для ўплыву на выбар аптымальнага маршруту і ўсе
неабходныя перавагі сучасных пратаколаў маршрутызацыі (хуткая збежнасць, адсутнасць петляў у сетцы і да таго падобнае).

\subsection{Прынцып работы пратакола EIGRP}

EIGRP (англ. Enhanced Interior Gateway Routing Protocol) --- пратакол маршрутызацыі, распрацаваны фірмай Cisco на аснове пратаколу IGRP той жа фірмы. Рэліз пратаколу адбыўся ў 1994 годзе. EIGRP выкарыстоўвае механізм DUAL (diffusing update algorithm) для выбару найбольш кароткага маршруту.

EIGRP больш просты ў рэалізацыі і меней патрабавальны да вылічальных рэсурсаў маршрутызатара чым пратакол OSPF (Open Shortest Path First). Таксама EIGRP мае больш удасканалены алгарытм вылічэння метрыкі (DUAL), які можа выкарыстоўваць 5 розных кампанентаў для разліку:

\begin{enumerate}
    \item прапускная здольнасць (Bandwidth) --- мінімальная прапускная здольнасць для дадзенага маршруту (а не сума коштаў (cost) у адрозненне ад OSPF).
    \item Затрымка (Delay) --- cумарная затрымка на ўсім шляху маршруту.
    \item Надзейнасць (Reliability) --- найгоршы паказчык надзейнасці на ўсім шляху маршруту, заснаваны на характарыстыкі keep alive.
    \item Нагрузка (Loading) --- найгоршы паказальнік нагрузкі інтэрфейса на ўсім шляху маршруту, заснаваны на колькасці трафіку, які праходзіць праз інтэрфейс і параметры bandwidth на гэтым інтэрфейсе.
    \item MTU (Maximum transmission unit) --- мінімальны памер MTU на ўсім шляху маршруту.
\end{enumerate}

Па ўмаўчанні, EIGRP выкарыстоўвае толькі першыя два кампаненты, так як надзейнасць і нагрузка --- дынамічныя велічыні, якія могуць змяняцца да некалькіх раз у секунду. Адпаведна, кожнае змяненне выклікае пераразлік метрыкі для маршрутаў і выкарыстанне працэсарнай магутнасці маршрутызатара да 100\%. MTU не з'яўляецца дынамічнай велічынёй, але не выкарыстоўваецца па прычыне слабога ўплыву на метрыку маршруту.

Асноўныя характарыстыкі EIGRP:
\begin{enumerate}
     \item хуткая збежнасць (у параўнанні з іншымі дыстанцыйна-вектарнымі пратаколамі);
     \item падтрымка VLSM (Variable Length Subnet Masking);
     \item частковыя абнаўлення паміж суседзямі;
     \item падтрымка розных пратаколаў сеткавага ўзроўня (IP, IPX, AppleTalk);
     \item аднолькавыя настройкі пратакола пры выкарыстанні розных пратаколаў канального ўзроўню (напрыклад, у OSPF настройкі адрозніваюцца для Ethernet і Frame Relay);
     \item складаная метрыка;
     \item выкарыстанне multicast (224.0.0.10) і unicast адрасоў, замест шырокавяшчальнай рассылкі;
\end{enumerate}

\subsubsection{RTP і тыпы паведамленняў EIGRP.}

RTP (Reliable Transport Protocol) кіруе працэсам адпраўкі і атрымання пакетаў EIGRP.

RTP забяспечвае:
\begin{enumerate}
    \item гарантаваную дастаўку пакетаў. Для гэтага выкарыстоўваецца прапрыетарны алгарытм Cisco, reliable multicast. Пакеты адпраўляюцца на multicast-адрас 224.0.0.10. Кожны сусед які атрымаў такі пакет адпраўляе пацверджанне адпраўніку пакета.
    \item захаванне парадку пакетаў. У кожным пакеце выкарыстоўваецца два нумары паслядоўнасці (sequence). Кожны пакет уключае ў сябе нумар прысвоены яму адпраўніком. Гэты нумар павялічваецца на адзінку кожны раз, калі маршрутызатар адпраўляе новы пакет. Акрамя таго, адпраўнік змяшчае ў пакет нумар апошняга атрыманага пакета ад атрымальніка.
    \item у некаторых выпадках RTP выкарыстоўвае негарантаваную дастаўку. У такіх пакетах нe прастаўляюцца нумары паслядоўнасці і яны не патрабуюць пацвярджэння аб атрыманнi.
\end{enumerate}

Усе паведамленні EIGRP інкапсулюецца ў IP-пакеты, нумар EIGRP ў полі пратакол IP-пакета --- 88.

EIGRP выкарыстоўвае 5 тыпаў паведамленняў:
\begin{enumerate}
    \item Hello --- маршрутызатары выкарыстоўваюць hello-пакеты для выяўлення суседзяў. Пакеты адпраўляюцца пры дапамозе multicast рассылкі і не патрабуюць пацвярджэння аб атрыманнi;
    \item Update --- змяшчаецца інфармацыя аб змене маршрутаў. Яны адпраўляюцца толькі маршрутызатарам, якіх тычыцца абнаўленне. Гэтыя пакеты могуць быць адпраўленыя канкрэтнаму маршрутызатару (unicast рассылка) або групе маршрутызатараў (multicast рассылка). Атрыманне update-пакета пацвярджаецца адпраўкай ACK пакета;
    \item Query --- калі маршрутызатар выконвае падлік маршруту і ў яго няма feasible successor, ён адпраўляе query-пакет сваім суседзям для таго каб вызначыць ці няма feasible successor для гэтага маршруту ў іх. Звычайна query-пакеты адпраўляюцца multicast рассылкай, але могуць быць і unicast рассылкай. Атрыманне query-пакета пацвярджаецца адпраўкай ACK атрымальнікам пакета;
    \item Reply --- маршрутызатар адпраўляе reply-пакет у адказ на query-пакет. Reply-пакеты адпраўляюцца unicast рассылкай таму, хто адправіў query-пакет. Атрыманне reply-пакета пацвярджаецца адпраўкай ACK;
    \item ACK --- пакет, які пацвярджае атрыманне пакетаў update, query, reply. ACK-пакеты адпраўляюцца unicast рассылкай і ўтрымліваюць у сабе acknowledgment number. Фактычна гэта hello-пакеты, якія не перадаюць даных. Выкарыстоўваецца негарантаваная дастаўка.
\end{enumerate}

\subsubsection{Адносіны суседства і пакеты Hello.}

Для ўстанаўлення адносін суседства EIGRP выкарыстоўвае пакеты hello:
\begin{enumerate}
    \item на ethernet-інтэрфейсах і point-to-point інтэрфейсах hello-пакеты па ўмаўчанні адпраўляюцца кожныя 5 секунд, але з невялікім выпадковым адхіленнем, якое выкарыстоўваецца для таго, каб паміж маршрутызатарамі не было сінхранізацыі ў адпраўцы hello-пакетаў;
    \item на multipoint X.25, Frame Relay, і ATM інтэрфейсах hello-пакеты адпраўляюцца unicast па змаўчанні кожныя 60 секунд;
    \item калі сусед не дасылае hello-паведамленне за перыяд hold time (па ўмаўчанні 15 секунд, 3 hello-інтэрвалы), то ён лічыцца недаступным.
\end{enumerate}

Для таго каб маршрутызатары сталі суседзямі павінны выконвацца такія ўмовы:
\begin{enumerate}
    \item маршрутызатары павінны прайсці аўтэнтыфікацыю;
    \item маршрутызатары павінны быць у адной AS (Autonomous System);
    \item адносіны суседства павінны ўстанаўлівацца на галоўных адрасах (калі прыходзіць hello-пакет, маршрутызатар правярае ці належыць адрас адпраўніка сетцы на галоўным адрасе інтэрфейса);
    \item павінны супадаць значэння K-каэфіцыентаў.
\end{enumerate}

Інфармацыя аб усіх выяўленых суседзях змяшчаецца ў табліцы суседзяў.
Табліца суседзяў (neighbor table) --- спіс непасрэдна далучаных маршрутызатараў (на якіх працуе EIGRP) з якімі маршрутызатар устанавіў адносіны суседства. Адна табліца суседзяў існуе для кожнага PDM (Protocol-dependent module).

\subsubsection{Update пакеты.}

Пасля таго як маршрутызатары сталі суседзямі, яны пачынаюць абменьвацца абнаўленнямі (Update). Гэтыя пакеты могуць быць адпраўленыя канкрэтнаму маршрутызатара (unicast рассылка) або групе маршрутызатараў (multicast рассылка).

Працэс абмену абнаўленнямі:
\begin{enumerate}
    \item першапачаткова адпраўляюцца поўныя абнаўлення, у якія ўключаны ўсе маршруты, за выключэннем тых, якія трапляюць пад правіла split horizon;
    \item пасля таго як абмен маршрутамі завяршыўся, абнаўленні не адпраўляюцца;
    \item у далейшым абнаўлення адпраўляюцца, калі змяніўся адзін або больш маршрутаў;
    \item калі адносіны суседства разрываюцца, а затым аднаўляюцца, то адпраўляюцца поўныя абнаўлення.
\end{enumerate}

Віды EIGRP абнаўленняў:
\begin{enumerate}
    \item неперыядычныя (Nonperiodic) --- абнаўленні адпраўляюцца нe праз рэгулярныя інтэрвалы часу, а пры змене тапалогіі або метрыкі;
    \item частковыя (Partial) --- у абнаўленнях перадаецца не ўся інфармацыя з табліцы маршрутызацыі, а толькі змены;
    \item абмежаваныя (Bounded) - абнаўленні адпраўляюцца толькі задзейнічаным маршрутызатарам.
\end{enumerate}

\subsubsection{Табліца тапалогіі}

Табліца тапалогіі (topology table) --- спіс маршрутаў, вывучаных ад кожнага суседа.

Калі сусед паведамляе лакальнаму маршрутызатару аб маршруце, то сусед павінен выкарыстоўваць гэты маршрут для перадачы трафіку. Іншымі словамі, маршрутызатар адпраўляе суседзям толькі тыя маршруты, якія сам выкарыстоўвае (гэта значыць, яны знаходзяцца ў табліцы маршрутызацыі).
Гэтае правіла абавязкова павінна выконвацца для ўсіх дыстанцыйна-вектарных пратаколаў.

У табліцы тапалогіі таксама захоўваецца метрыка, якую паведамляе кожны сусед для кожнага маршруту (AD) і метрыка, якую лакальны маршрутызатар будзе выкарыстоўваць для таго каб дасягнуць маршруту праз суседа.

Табліца тапалогіі абнаўляецца, калі змяняецца непасрэдна далучаны маршрут або інтэрфейс або калі суседні маршрутызатар паведамляе пра змяненне маршруту.
Запісы ў табліцы тапалогіі могуць знаходзіцца ў двух станах: active і passive.
Маршрут знаходзіцца ў стане passive, калі маршрутызатар не выконвае пералік маршруту, і ў стане active --- калі выконваецца пералік маршруту.

Пералік выконваецца, калі для сеткі прызначэння няма feasible successor. Маршрутызатар ініцыюе пералік адпраўляючы запыт (адпраўляе query packet) кожнаму суседняму маршрутызатару. Калі ў суседа ёсць маршрут да сеткі прызначэння, то ён адказвае (адпраўляе reply packet), калі маршруту няма --- сусед адпраўляе запыт сваіх суседзяў.

У табліцы тапалогіі можа захоўваецца 6 маршрутаў да сеткі прызначэння (асноўны і запасныя).

\subsubsection{Разлік метрыкі EIGRP.}

У стандарце, які апісвае работу EIGRP, вызначаны 2 варыянты метрыкі:
стандартная і пашыраная. Стандартная метрыка, якая будзе выкарыстоўвацца ў дадзеным праекце, разлічваецца маршрутызатарам
па формуле:
\begin{equation}
    CM = \left(K_1 \cdot BW_s + K_2 \cdot \frac{BW_s}{256 -Lo_\text{max}} + K_3 \cdot D_s\right) \cdot \left(\frac{K_5}{K_4 + R_\text{min}}\right)
\end{equation}
\begin{Explanation}
    \item[дзе] $BW_s$ = $\frac{256 \cdot 10^7}{Bandwidth_\text{min}}$,
    $Lo_\text{max}$ -- нагрузка,
    $D_s$ -- $256 \cdot Delay_\text{сум}$,
    $R$ -- надзейнасць.
\end{Explanation}

Як бачна з формулы, пры разліку метрыкі для канкрэтнага прэфікса
выкарыстоўваецца параметр Load (нагрузка), максімальнае яго значэнне па ўсіх кірунках на маршруце да зададзенага прэфікса.
Каэфіцыенты $K_1$, $K_2$, $K_3$, $K_4$, $K_5$ ўяўляюць сабой вагавыя
каэфіцыенты, якія змяняюцца ад 0 да 255, яны служаць для таго, каб падчас разліку метрыкі
той ці іншы параметр аказваў большы ці меншы ўплыў на канчатковы
вынік разліку. Верхні парог --- 255 абумоўлены неабходнасцю маштабавання метрыкі да памеру ў 32 біты --- максімальнага памеру значэння
метрыкі ў табліцы маршрутызацыі. Ва ўсіх бягучых рэалізацыях EIGRP значэння
каэфіцыентаў: $K_1 = 1$, $K_2 = 0$, $K_3 = 1$, $K_4 = 0$, $K_5 = 0$, такім чынам рэальную ролю пры разліку
метрыкі гуляюць толькі затрымка і прапускная здольнасць.

Пры ўключэнні ў разлік параметра Load, гэта значыць устаноўка значэння
каэфіцыента $K_2 > 1$ гэты параметр ўлічваецца толькі пры першапачатковым
разліку метрыкі для маршруту. Пасля першага разліку і атрымання значэння
метрыкі нават пры змене параметру Load гэтыя змены не ўлічваюцца
і пераразлік метрыкі не вырабляецца. Дадзеная асаблівасць звязаная з наступнымі складанасцямі. Як вядома з палажэнняў тэорыі масавага абслугоўвання, нагрузка (трафік), якія генерыруюцца падлучанымі да сеткі прыладамі,
мае імавернасную прыроду, гэта значыць значэнне нагрузкі, якая прысутнічае ў сетцы
не пастаянная і змяняецца ў часе выпадковым чынам. размеркаванне імавернасці нагрузкі трафіку пры гэтым адрозніваецца, у залежнасці
ад тыпу сeрвісаў, якія прадстаўляюцца сеткай і тыпу прылад, падлучаных
да сеткі. У выніку, змена параметра Load, які адлюстроўвае змяненне
нагрузкі, якая дзейнічае на інтэрфейсе маршрутызатара, мае нелінейную
прыроду; змена таксама адбываецца ў адпаведнасці з дзеючым у дадзенай
сетцы законам размеркавання верагоднасцяў, з-за чаго значэнне параметру можа моцна адрознівацца па абсалютнай велічыні значэння на
кароткім інтэрвале часу. У выніку, часты пераразлік метрыкі з рознымі значэннямі параметру Load можа прыводзіць да частага
змянення маршруту для праходжання трафіку да зададзенага прэфікса, што:
\begin{enumerate}
    \item павялічвае затрымку пры перадачы трафіку;
    \item прыводзіць да страты некаторых пакетаў ў моманты нестабільнасці сеткі;
    \item выклікае змены ў паслядоўнасці перадаюцца пакетаў.
\end{enumerate}

Акрамя таго, змена метрыкі маршрута выклікае работу механізмаў
дыфузійных вылічэнняў, закладзеныя ў EIGRP, з-за чаго, інфармацыя
аб частых зменах метрыкі, і, адпаведна, маршрута, перадаецца
іншым маршрутызатарам, якія ўдзельнічаюць у працы EIGRP, якія таксама пералічваюць
метрыку і змяняюць рашэннe аб маршрутызацыі, што, у выніку, прыводзіць да
пастаяннага змянення ўсіх маршрутаў і нестабільнасці сеткі,
у выніку якіх негатыўныя фактары, пералічаныя вышэй, узмацняюцца
і павялічваецца колькасць адмоў у абслугоўванні, што адбіваецца на даступнасці інфармацыі, якая перадаецца па сетцы.
Па ўсіх названых прычынах, як было адзначана вышэй, пасля першага
разліку і атрымання значэння метрыкі нават пры змене параметру Load гэтыя
змены не ўлічваюцца і пераразлік метрыкі не робіцца.

\subsection{Алгарытм пераразліку нагрузкі}

Бягучая рэалізацыя пратакола не выкарыстоўвае параметры Load для разліку метрыкі. Для таго, каб абыйсці гэтыя абмежаванні, можна выкарыстоўваць
адаптыўны алгарытм рэагавання на змены нагрузкі, які задаецца рознасным раўнаннем:
\begin{equation}
    Load = \alpha \cdot Load + (1 - \alpha) \cdot Load_\text{new}\text{, } 0 <= \alpha <= 1.
\end{equation}

Згодна з асаблівасцю рознаснага раўнення, значэнне параметра Load,
якое вылічваецца за бягучы інтэрвал на кантролеры, будзе залежаць ад значэння
параметра Load, вылічанага на папярэднім інтэрвале і значэння параметра Load, атрыманага за бягучы інтэрвал ад маршрутызатара. Пры гэтым вага
апошняга ў агульным выніку вылічэнняў бягучага такту будзе залежаць ад
значэння каэфіцыента $\alpha$. З павелічэннем дадзенага каэфіцыента памяншаецца адчувальнасць дадзенага алгарытму да зменаў у нагрузцы, з памяншэннем дадзенага каэфіцыента павялічваецца адчувальнасць алгарытму
да зменаў у нагрузцы.

Пытанне аб тым, якое значэнне каэфіцыента выбраць
у выпадку канкрэтных тапалогіі і мадэлі трафіку (закона размеркавання імавернасці нагрузкі) з'яўляецца адкрытым і патрабуе далейшага даследавання.

На ўзроўні кантролера задаецца парогавае значэнне, якое забяспечвае ўмовы рэакцыі на змены ў нагрузцы, сумесна з заданнем каэфіцыента $\alpha$ вызначаецца агульная рэакцыя алгарытму на змены нагрузкі ў сетцы.
Паколькі пры вылічэнні метрыкі для маршруту маршрутызатарам выкарыстоўваюцца цэлыя значэння параметраў (прапускной здольнасці, затрымкі,
нагрузкі, надзейнасці), то і вылічэнні на ўзроўні кантролера мае сэнс
вырабляць з цэлымі лікамі, тады вынік вылічэнняў варта акругляць
да бліжэйшага цэлага значэння, каб пасля перадаваць маршрутызатару.
Вылічэнні робяцца для кожнага інтэрфейса маршрутызатара, параметры якога можа атрымаць кантролер.

Пасля таго, як з прычыны змены значэння нагрузкі было дасягнута пэўнае (зададзенае адміністратарам) парогавае значэнне гэтага параметра, кантролер перадае на маршрутызатары (адзін або некалькі) рашэнне
аб пераразліку маршрута. Паколькі маршрутызатар, які ўдзельнічае ў працы
EIGRP, вядомы бягучыя значэнні параметра нагрузкі, то перадача гэтых параметраў не патрабуецца. Пасля атрымання рашэння аб пераразліку маршрутаў
маршрутызатар запускае вылічэнні метрыкі для ўсіх маршрутаў, або для тых
маршрутаў, якія змяніліся значэннем метрыкі, згодна
бягучых спецыфікацыям EIGRP без усялякіх зменаў. Такая канчатковая рэалізацыя прапанаванага метаду задзейнічае ўжо наяўныя на сеткавых прыладах механізмы і алгарытмы і не патрабуе зменаў ні ў апаратнай, ні
ў праграмнай рэалізацыі сеткавых прылад. Прапанаваны алгарытм разам
з усімі вылічэннямі разгортваецца на кантролеры, у ролі якога можа
выступаць любая платформа, апаратная або праграмная з падтрымкай адпаведных праграмных інтэрфейсаў.

На малюнку \ref{img: block-schema} прадстаўлена блок-схема алгарытму пераразліку нагрузкі.

\clearpage

\begin{figure}[h!]
    \centering
    \includegraphics[width=0.55\linewidth]{block-schema.pdf}
    \caption{Блок-схема алгарытму пераразліку нагрузкі}
    \label{img: block-schema}
\end{figure}

\subsection{Механізм збору інфармацыі з сеткавага вузла}

Для атрымання інфармацыі з сеткавага вузла могуць выкарыстоўвацца такія тэхналогіі, як:
\begin{enumerate}
    \item SNMP пратакол;
    \item NETCONF/YANG пратакол;
    \item RESTCONF пратакол.
\end{enumerate}

Коратка разгледзім кожную з іх.

SNMP (англ. Simple Network Management Protocol) --- стандартны інтэрнэт-пратакол для кіравання прыладамі ў IP-сетках на аснове архітэктур TCP/UDP. Пратакол звычайна выкарыстоўваецца ў сістэмах сеткавага кіравання для кантролю падлучаных да сеткі прылад на прадмет умоў, якія патрабуюць увагі адміністратара. SNMP вызначаны Інжынерным саветам інтэрнэту (IETF) як кампанент TCP/IP. Ён складаецца з набору стандартаў для сеткавага кіравання, уключаючы пратакол прыкладнога ўзроўню, схему баз даных і набор аб'ектаў даных.

Пры выкарыстанні SNMP адзін альбо некалькі адміністрацыйных камп'ютараў выконваюць адсочванне або кіраванне групай хастоў або прылад у камп'ютарнай сетцы. На кожнай кіраванай сістэме ёсць пастаянна запушчаная праграма, агент, якая праз SNMP перадае кіруючую інфармацыю.

Кіраваныя пратаколам SNMP сеткі складаюцца з трох ключавых кампанентаў:
\begin{enumerate}
    \item кіраванае ўстройства;
    \item агент --- праграмнае забеспячэнне, якое запускаецца на кіраванай прыладзе, альбо на прыладзе, падлучанай да інтэрфейса кіравання кіраванай прылады;
    \item сістэма сеткавага кіравання (Network Management System, NMS) --- праграмнае забеспячэнне для падтрымкі комплекснай структуры даных, якая адлюстроўвае стан сеткі.
\end{enumerate}

Кіраванае ўстройства--- элемент сеткі (абсталяванне або праграмны сродак), які рэалізуе інтэрфейс кіравання (не абавязкова SNMP), які дазваляе аднанакіраваны (толькі для чытання) або двухнакіраваны доступ да канкрэтнай інфармацыі аб элеменце. Кіраваныя ўстройства абменьваюцца гэтай інфармацыяй з кантролерам.

Агентам называецца праграмны модуль сеткавага кіравання, які размяшчаецца на кіраваным устройстве, альбо на прыладзе, падлучанай да інтэрфейса кіравання. Агент валодае лакальным веданнем кіруючай інфармацыі і перакладае гэтую інфармацыю ў спецыфічную для SNMP форму.

NMS забяспечваюць асноўную частку апрацоўкі даных, неабходных для сеткавага кіравання. У любой кіраванай сеткі можа быць адна і больш NMS.

SNMP працуе на прыкладным узроўні TCP/IP (сёмы ўзровень мадэлі OSI). Агент SNMP атрымлівае запыты па UDP-порту 161. Кантролер можа пасылаць запыты з любога даступнага парта крыніцы на порт агента. Адказ агента будзе адпраўлены назад на порт крыніцы на кантролер. Кантролер атрымлівае паведамленні па парту 162. Агент можа генераваць апавяшчэнні з любога даступнага парта. Пры выкарыстанні TLS або DTLS запыты атрымліваюцца на парту 10161, а trap-паведамленні адпраўляюцца на порт 10162.

Сеткавы пратакол канфігурацыі, NETCONF, з'яўляецца пратаколам сеткавага кіравання ўстройств. Ён быў распрацаваны ў рамках працоўнай групы NETCONF і ўпершыню апублікаваны ў RFC 4741, які быў перапрацаваны ў RFC 6241 у чэрвені 2011.

NETCONF прадастаўляе механізмы ўстаноўкі, кіравання і выдалення канфігурацыі сеткавых прылад з дапамогай аддаленага выкліку працэдур RPC. NETCONF выкарыстоўвае XML як сродак прадастаўлення канфігурацыі і як сродак фарміравання паведамленняў пратакола, які рэалізуецца па-над транспартным узроўні.

NETCONF можа быць схематычна падзелены на чатыры ўзроўні, якія паканы па малюнку \ref{img: netconf}.

\clearpage

\begin{figure}[ht!]
    \centering
    \includegraphics[width=0.7\textwidth]{netconf.pdf}
    \caption{Чатыры ўзроўні NETCONF}
    \label{img: netconf}
\end{figure}

У табліцы \ref{table: netconf} прадстаўлена параўнанне пратаколаў SNMP і NETCONF.

\clearpage

\begin{table}[htp]
    \caption{Параўнанне пратаколаў кіравання SNMP і NETCONF}
    \begin{tabularx}{\textwidth}{ 
        | >{\centering\arraybackslash}X 
        | >{\centering\arraybackslash}X 
        | >{\centering\arraybackslash}X |
    }
    \hline
    Сцэнарый & SNMP & NETCONF \\
    \hline
    Атрыманне групы параметраў аб стане & Так & Так (нашмат хутчэй, чым SNMP) \\
    \hline
    Змяненне групы параметраў &  Так (да 64 КБайт) & Так \\
    \hline
    Транзакцыйнае змяненне параметраў & Не & Так \\
    \hline
    Транзакцыя аперацыі на групе сеткавых устройств & Не & Так \\
    \hline
    Выклік адміністратыўных функцый & Тэарэтычна & Так \\
    \hline
    Адпраўка паведамленняў & Так & Так \\
    \hline
    Рэзервнае капіраванне і аднаўленне & Звычайна не & Так \\
    \hline
    Пратакол абароны & V3 & Так \\
    \hline
    Тэсціраванне канфігурацыі перад прымяненнем & Не & Так \\
    \hline
    \end{tabularx}
    \label{table: netconf}
\end{table}

З табліцы \ref{table: netconf} можам зрабіць вынік, што ў сучасных тэлекамунікацыйных сетках
больш пажадана выкарыстоўваць пратакол кіравання NETCONF, так як ён мае значныe ўдасканальванне
ў параўнанні з пратаколам кіравання SNMP.

Асобна разгледзім пратакол кіравання RESTCONF.

RESTCONF аб'ядноўвае прастату HTTP з прадказальнасцю і магчымасцямі аўтаматызацыі кіраванымі схемамі API. Ведаючы модулі YANG, якія выкарыстоўваeцца сервер, кліент можа вывесці URL ўсіх рэсурсаў кіравання, а таксама прыдатную структуру ўсіх запытаў і водгукаў RESTCONF.

RESTCONF выкарыстоўвае бібліятэку NETCONF/YANG для прадастаўлення кліентам магчымасці знайсці інфармацыю у адпаведнасці з модулем YANG на сeрвэры, калі кліент хоча скарыстацца ёй.

Ідэнтыфікатары URI для залежных ад мадэлі даных аперацый RPC і змесціва сховішчаў прадказальныя на аснове азначэнняў у модулі YANG.

RESTCONF працуе з канцэптуальным сховішчам даных, вызначаным на мове мадэлявання YANG. Сервер пералічвае ўсе модулі YANG з выкарыстаннем модуля ietf-yang-library. Сервер павінен рэалізаваць модуль ietf-yang-library, які павінен паказваць усе выкарыстоўваныя серверам модулі YANG ў спісе modules-state/module. Змесціва канцэптуальнага сховішчы даных, залежныя ад мадэлі аперацыі RPC і апавяшчэння аб падзеях таксама паказваюцца гэтым наборам модуляў YANG.

Класіфікацыя даных якія адносяцца ці якія не адносяцца да канфігурацый выводзіцца з аператара YANG config. Паводзіны, звязанае з парадкаваннем даных, вызначаецца аператарам YANG ordered-by. Даныя, якія не адносяцца да канфігурацыі, называюць таксама данымі стану.

Мадэль рэдагавання сховішчы даных у RESTCONF простая і прамалінейная, падобна паводзінам магчымасці :writable-running ў NETCONF. Кожнае рэдагаванне RESTCONF рэсурсу даных у рэсурсе сховішчы актывіруецца пасля паспяховага завяршэння рэдагавання.

На малюнку \ref{img: netconf-restconf} прадстаўлена ўзаемадзеянне мадэлі данных YANG з пратаколамі кіравання.

\begin{figure}[ht!]
    \centering
    \includegraphics[width=0.7\textwidth]{netconf_restconf.pdf}
    \caption{Узаемадзеянне мадэлі данных YANG з пратаколамі кіравання}
    \label{img: netconf-restconf}
\end{figure}

\subsection{Рэалізацыя механізму пераразліку нагрузкі}

Праграма, якая рэалізуе вышэй апісаны алгарытм уліку нагрузкі, напісана
на аб'ектна-арыентаванай мове праграмавання Python. Выбар у якасці мовы праграмавання
Python абумоўлены яго шматплатформеннасцю, а таксама вялікай колькасцю модуляў і бібліятэк, якія пашыраюць базавы функцыянал.

Праграма складаецца з 3 асноўных частак:
\begin{enumerate}
    \item Recalculation --- клас, які адказвае за працэс маніторынгу параметраў маршрутызатара і прыняцця рашэння аб
    пераразліку тапалогіі сеткі;
    \item Router --- клас, які ўяўляе сабой апісанне маршртузатара, і рэалізуе
    метады збору і перадачы інфармацыі на маршрутызатар;
    \item Interface --- клас, які ўяўляе сабой апісанне інтэрфейса маршрутызатара,
    і рэалізуе метады збору інфармацыі з дадзенага інтэрфейса.
\end{enumerate}

На малюнку \ref{uml: Class Diagram} прадстаўлена дыяграма класаў.

\clearpage

\begin{figure}[ht!]
    \centering
    \includegraphics[width=\textwidth]{class_diagram.png}
    \caption{Дыяграма класаў праграмы пераліку нагрузкі}
    \label{uml: Class Diagram}
\end{figure}

\vspace{-\baselineskip}

\subsubsection{Апісанне класса Interface.}

Дадзены клас змяшчае ўсю неабходную інфармацыю пра інтэрфейс маршрутызатара і
рэалізуе метады для работы з ім.

Атрыбуты класа:
\begin{enumerate}
    \item name --- імя інтэрфейса на маршрутызатары (напрыклад, GigabitEthernet1);
    \item txload --- значэнне нагрузкі перадачы на інтэрфейсе.
\end{enumerate}

Метады класа:
\begin{enumerate}
    \item get\_name() --- вяртае імя інтэрфейса;
    \item get\_txload() --- вяртае значэнне нагрузкі перадачы на інтэрфейсе;
    \item set\_txload(txload) --- выстаўляе значэнне нагрузкі перадачы на інтэрфейсе
    роўным txload.
\end{enumerate}

У лістынгу \ref{lst: interface} прадстаўлены зыходны код Interface класа.

\lstinputlisting[caption={Зыходны код Interface класа},%
                            label={lst: interface},%
                            language=Python]{interface.py}

\vspace{-\baselineskip}

\subsubsection{Апісанне класа Router.}

Дадзены клас змяшчае ўсю неабходную інфармацыю пра маршрутызатар і рэалізуе
метады для работы з ім.

Атрыбуты класа:
\begin{enumerate}
    \item host --- IP-адрас маршрутызатара, на які будзе ажыццяўляцца падключэнне кантролерам;
    \item username --- імя карыстальніка для аўтэнтыфікацыі і аўтарызацыі;
    \item password --- пароль карыстальніка для аўтэнтыфікацыі і аўтарызацыі;
    \item interfaces --- спіс усіх інтэрфейсаў на маршрутызатары.
\end{enumerate}

Метады класа:
\begin{enumerate}
    \item get\_host() --- вяртае IP-адрас маршрутызатара;
    \item \_set\_interfaces() --- збірае і ўстанаўлівае інфармацыю пра інтэрфейсы на маршрутызатары;
    \item get\_interfaces() --- вяртае ўсе інтэрфейсы маршрутызатара;
    \item get\_interface\_names() --- вяртае ўсе назвы інтэрфейсаў;
    \item get\_interface\_txload(interface) --- вяртае нагрузку перадачы інтэрфейса;
    \item restart\_eigrp() --- перазагружае eigrp пратакол для пералічэння сеткавай тапалогіі;
    \item \_disable\_eigrp() --- адключае eigrp пратакол;
    \item \_enable\_eigrp() --- уключае eigrp пратакол.
\end{enumerate}

У лістынгу \ref{lst: router} прадстаўлены зыходны код Router класа.

\lstinputlisting[caption={Зыходны код Router класа},%
                            label={lst: router},%
                            language=Python]{router.py}

\vspace{-\baselineskip}

\subsubsection{Апісанне класа Recalculation.}

Дадзены клас рэалізуе механізм маніторынгу і прыняцця рашэння для пераразліку тапалогіі сеткі на маршрутызатары.
Клас Recalculation унаследаваны ад класа Thread (модуль threading) для рэалізацыі паралельнай работы кантролера для
некалькіх маршрутызатараў: для кожнага маршрутызатара выкарыстоўваецца асобны паток.

Атрыбуты класа:
\begin{enumerate}
    \item TIMEOUT --- прамежак часу (у секундах) праз які запрашваецца значэнне нагрузкі з інтэрфейсаў маршрутызатара;
    \item coefficient --- значэнне каэфіцыента $\alpha$;
    \item threshold --- парог рознасці паміж новай і старой нагрузкай, пры
    перавышэнні якога адпраўляецца каманда на перабудову сеткавай тапалогіі.
\end{enumerate}

Метад класа:
\begin{enumerate}
    \item run() --- перазапіс метада класа Thread для паралельнага выконвання праграмы. Выконвае маніторынг і пералік нагрузкі для маршрутызатараў, інфармацыя пра якіх была перададзена ў метад.
\end{enumerate}

У лістынгу \ref{lst: recalculation} прадстаўлены зыходны код Recalculation класа.

\lstinputlisting[caption={Зыходны код Recalculation класа},%
                            label={lst: recalculation},%
                            language=Python]{recalculation.py}

У лістынгу  \ref{lst: main} прадстаўлены прыклад запуску праграмы для адсочвання
нагрузкі і пералічэння сеткавай тапалогіі для 1 маршрутызатара.

\lstinputlisting[caption={Зыходны код галоўнай праграмы для 1 маршрутызатара},%
                            label={lst: main},%
                            language=Python]{main.py}

