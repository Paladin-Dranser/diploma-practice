% Метод управления трафиком в гибридной программно-определяемой сети
\section{Элементы праграмнага мадэліраванне размеркавання патокаў}

На аснове вышэй апісаных вывадаў разгледзем гібрыдную мадэль SDN
на аснове пратакола маршрутызацыі EIGRP.

Дадзены пратакол адносіцца да тыпу distance-vector, аднак, у адрозненне ад
RIP, ён выкарыстоўвае ў якасці асновы сваёй работы алгарытм дыфузійных вылічэнняў.

Маршрутызатары, якія ўдзельнічаюць у працы EIGRP абменьваюцца інфармацыяй аб прэфіксах, якія знаходзяцца ў табліцы маршрутызацыі кожнага з іх.

Інфармацыя аб прэфіксах змяшчае:
\begin{enumerate}
    \item IP-адрас сеткі;
    \item маску падсеткі;
    \item паласу прапускання сегмента гэтай сеткі;
    \item затрымку на інтэрфейсе маршрутызатара, які ўяўляе дадзены сегмент сеткі;
    \item нагрузку, якая дзейнічае на інтэрфейсе, які ўяўляе дадзены сегмент сеткі;
    \item надзейнасць, разлічаная на інтэрфейсе дадзенага сегмента сеткі;
    \item памер MTU.
\end{enumerate}

У якасці паласы прапускання сегмента сеткі перадаецца мінімальная
паласа прапускання па сеткаваму шляху, праз які праходзіць маршрут да зададзенай сеткі. У якасці затрымкі перадаецца сумарнаe значэнне затрымкі
для перадачы пакета па ўсіх каналах на маршруце да зададзенай сеткі. У якасці нагрузкі перадаецца максімальнае значэнне параметру txload па ўсіх каналах на маршруце да зададзенай сеткі, які аўтаматычна разлічваецца
на кожным інтэрфейсе маршрутызатара. У якасці надзейнасці перадаецца мінімальнае
значэнне адносін колькасці дакладна прынятых пакетаў да агульнай колькасці прынятых пакетаў па ўсіх каналах на маршруце да зададзенай сеткі. Значэнні
нагрузкі і надзейнасці вылічаюцца кожным маршрутызатарам паасобку
на інтэрфейсе, падключаным да канала, які з'яўляецца часткай маршруту
да зададзенай сеткі.
