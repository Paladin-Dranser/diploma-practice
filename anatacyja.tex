\sectionWithoutNumber{АННОТАЦИЯ}

Данная дипломная работа посвящена программному моделированию
распределения потоков трафика в программно-определяемых сетях.

В первом разделе рассмотрены определение программно-определяемых
сетей, архитектура программно-определяемых сетей, подходы к реализации
концепции программно-определяемых сетей, поставлена задача распределения
потоков трафика в компьютерных сетях.

Во втором разделе выполнен сравнительный анализ протоколов маршрутизации
и их применение в программно-определяемых сетях, разработан алгоритм
перерасчёта нагрузкі на интерфейсах сетевого устройства, описан механизм
управления сетевым устройством, разработано программное обеспечение для
распределения потоков трафика в зависимости от нагрузки в программно-определяемых сетях.

В третьем разделе приведён пример моделирования программно-определяемой сети гибридной модели из 3 маршрутизаторов, дано сравнение параметров
маршрутизаторов до начала передачи информации и во время передачи информации в сети.

В четвёртом разделе выполнен экономический расчёт разработки программного
обеспечения для контролера в программно-определяемых сетях гибридной модели.

В пятом разделе показаны правила и рекомендации для создания эргономичных
условий труда.

Объём пояснительной записки составляет --- листов и содержит
--- таблиц, --- рисунком, --- формул, --- источников литературы.
