\sectionWithoutContent{Анатацыя}
\thispagestyle{empty}

Дадзеная дыпломная праца прысвечана праграмнаму мадэліраванню
размеркавання патокаў трафіка ў праграмна-вызначаных сетках.

У першым раздзеле разгледжаны азначэнне праграмна-вызначаных сетак,
архітэктура праграмна-вызначаных сетак, падыходы да рэалізацыі
канцэпцыі праграмна-вызначаных сетак, пастаўлена задача
размеркавання патокаў трафіка ў кап'ютарных сетках.

У другім раздзеле выканан параўнальны аналіз пратаколаў маршрутызацыі
і іх прымяненне ў праграмна-вызначаных сетках, распрацаваны алгарытм
пераразліку нагрузкі на інтэрфейсах сеткавага ўстройства,
апісаны механізм кіравання сеткавым устройствам, распрацавана
праграмнае забеспячэнне для размеркавання патокаў трафіка ў
залежнасці ад нагрузкі ў праграмна-вызначаных сетках.

У трэцім раздзеле прыведзены прыклад мадэліравання праграмна-вызначанай
сеткі гібрыднай мадэлі з 3 маршрутызатараў, дадзена параўнанне параметраў
маршрутызатараў да пачатку перадачы інфармацыі і ў час перадачы інфармацыі
ў сетцы.

У чацвёртым раздзеле выканан эканамічны разлік распрацоўкі праграмнага
забеспячэння для кантролера ў праграмна-вызначаных сетках гібрыднай мадэлі.

У пятым раздзеле паказаны правілы і рэкамендацыі для стварэння
эрганамічных умоў працы.

Аб'ём тлумачальнай запіскі складае -- аркушаў і змяшчае
-- табліц, -- малюнкаў, -- формул, -- крыніц літаратуры.

\clearpage

%\sectionWithoutNumber{Анатацыя}
\begin{center}
\fontsize{14}{14}
\selectfont
\textbf{АННОТАЦИЯ}
\end{center}
\thispagestyle{empty}

Данная дипломная работа посвящена программному моделированию
распределения потоков трафика в программно-определяемых сетях.

В первом разделе рассмотрены определение программно-определяемых
сетей, архитектура программно-определяемых сетей, подходы к реализации
концепции программно-оп\-ре\-де\-ля\-емых сетей, поставлена задача распределения
потоков трафика в компьютерных сетях.

Во втором разделе выполнен сравнительный анализ протоколов маршрутизации
и их применение в программно-определяемых сетях, разработан алгоритм
перерасчёта нагрузкі на интерфейсах сетевого устройства, описан механизм
управления сетевым устройством, разработано программное обеспечение для
распределения потоков трафика в зависимости от нагрузки в программно-определяемых сетях.

В третьем разделе приведён пример моделирования программно-определяемой сети гибридной модели из 3 маршрутизаторов, дано сравнение параметров
маршрутизаторов до начала передачи информации и во время передачи информации в сети.

В четвёртом разделе выполнен экономический расчёт разработки программного
обеспечения для контролера в программно-определяемых сетях гибридной модели.

В пятом разделе показаны правила и рекомендации для создания эргономичных
условий труда.

Объём пояснительной записки составляет -- листов и содержит
-- таблиц, -- рисунком, -- формул, -- источников литературы.
