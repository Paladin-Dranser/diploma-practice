\section{Канцэпцыя праграмна-вызначаных сетак}

% УЗЯТА З ARTICLE_1 (шукай малюнкі там)
З'яўленне сеткі Інтэрнэт прывяло да стварэння лічбавага грамадства, у якім
амаль усе прылады злучаны паміж сабой і даступны з любога пункту свету.
Аднак нягледзячы на сваю распаўсюджанасць, сеткі IP маюць складаную структуру, і імі
давалі складана кіраваць. Такую сетку складана сканфігураваць для дасягнення выбранай
сеткавай палітыкі, а таксама рэканфігураваць у адпаведнасці са знойдзенымі недахопамі і
змяненнем нагрузкі і структуры. Да ўскладнення сітуацыі прыводзіць і тое, што сеткі,
якія ўжо існуюць, з'яўляюцца вертыкальна інтэграванымі: узровень кіравання сеткай і
ўзровень патока даных злучаны ў адзін.

Праграмна-вызначаная сетка (SDN) --- гэта канцэпт, які дае надзею на змяненне сітуацыі.
У такой сетцы адмаўляюцца ад вертыкальнай інтэграцыі, выносяць кантрольную логіку
за межы камутатараў і маршрутызатараў, а таксама ўносяць магчымасць праграмаваць сетку.
SDN спрашчае працэс стварэння і ўкаранення новых сеткавых абстракцый, пры гэтым
спрашчаючы кіраванне сеткай, і дапамагае эвалюцыі сеткі.

Каб апісаць высокаўзроўневыя палітыкі сеткі, аператары сеткі вымушаны канфіругаваць кожную
прыладу асобна, карыстаючыся нізкаўзроўневымі камандамі, якія часта адрозніваюцца ў
залежнасці ад вытворцы абсталявання. Акрамя таго, сеткавыя структуры павінны
вытрымліваць няспраўнасці і адаптавацца да змяненням узроўня нагрузкі.
Механізмаў аўмататычнай рэканфігурацыі практычна не існуе пры бягучым стане IP сетак.
Праз гэта ўскладняецца ўкараненне сеткавых палітык у такім пастаянна зменлівым асяроддзі.

Сучасныя сеткі таксама вертыкальна інтэграваныя, што толькі ўскладняе сітуацыю.
Узровень кіравання (які прымае рашэнне наконт апрацоўкі сеткавага трафіка) і
ўзровень даных (які накіроўвае трафік адпаведна рашэнням узроўня кіравання) звязаны
ўнутры сеткавай прылады, што змяншае гібкасць і перашкаджае інавацыям і эвалюціі
сеткавай інфраструктуры.

Праграмна-вызначаная сетка з'яўляецца канцэпцыяй, якая яшчэ развіваецца, якая дае надзею
на змяненне абмежаванняў сучасных сеткавых структур.
Па-першае, яна разбівае вертыкальную інтэграцыю, гэта значыць адасабляе логіку кіравання
(узровень кіравання) ад маршрутызатараў і камутатараў, якія накіроўваюць трафік (узровень даных).
Па-другое, з падзелам гэтых узроўняў, камутатары становяцца простымі прыладамі для маршрутызацыі,
а кіраванне сканцэнтравана ў лагічным цэнтралізаваным кантролеры (сеткавая аперацыйная сістэма),
што спрашчае ўвядзенне сеткавых палітык, рэканфігурацыю сеткі і яе эвалюцыю.
Спрошчаны від такой архітэктуры паказаны на малюнку 1.

\begin{center}
УСТАВІЦЬ МАЛЮНАК
\end{center}

Трэба адзначыць, што лагічна цэнтралізаваня мадэль не азначае фізічна цэнтралізаваную сістэму.
Насамрэч, для забеспячэння адпаведных узроўняў хуткадзейнасці, маштабавання і надзейнасці,
ад таіх рашэнняў трэба адмовіцца. Замест гэтага, SDN сеткі апіраюцца на фізічна размеркаваныя
ўзроўні кіравання.

\subsection{Архітэктура праграмна-вызначаных сетак}
