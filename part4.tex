\section{Ахова працы}

\subsection{Арганізацыя сістэмы кіравання аховай працы на прадпрыемстве}

Сістэма кіравання аховай працы (АП) у замежным таварыстве з абмежаванай адказнасцю «ЭПАМ Сістэмз», прызначана для прадухілення траўм і захворванняў, звязаных з вытворчай дзейнасцю, а таксама для абароны і ўмацавання здароўя супрацоўнікаў арганізацыі; яна таксама накіравана на паляпшэнне ўмоў працы і навакольнага асяроддзя.

На малюнку \ref{img: epam healthy system} прадстаўлена структура службы аховы працы ў арганізацыі «ЭПАМ Сістэмз».

\begin{figure}[h!]
    \centering
    \includegraphics[width=0.7\textwidth]{epam_healthy_system.pdf}
    \caption{Структура службы аховы працы ў арганізацыі}
    \label{img: epam healthy system} 
\end{figure}

Штогод, у мэтах далейшага ўдасканалення арганізацыі работ па АП і забеспячэння выканання заканадаўства аб АП, генеральны дырэктар сцвярджае план мерапрыемстваў, складзены дырэктарам адміністрацыйнага дэпартамента і начальнікам глабальнага аддзела АП, якi ўключае:
\begin{enumerate}
    \item удасканаленне дакументацыі па АП, неабходнай у арганізацыі, ўкараненне перадавога досведу і навуковых распрацовак па АП і ТБ у арганізацыі;
    \item правядзенне папярэдніх медыцынскіх аглядаў, кансультацыі з дактарамі вузкіх спецыялізацый і вакцынацыя супрацоўнікаў перад прагназуемымі перыядамі эпідэміялагічных захворванняў;
    \item набыццё мыйных сродкаў і сродкаў асабістай гігіены для супрацоўнікаў арганізацыі, закупка сродкаў індывідуальнай абароны;
    \item арганізацыя правядзення лабараторна-інструментальных даследаванняў фактараў вытворчага асяроддзя на працоўных месцах у арандаваных памяшканнях;
    \item ажыццяўленне кантролю за правядзеннем перыядычных (паўторных) інструктажоў па АП і пажарнай бяспекі (ПБ) у вытворчых падраздзяленнях арганізацыі;
    \item выяўленне вытворчых небяспекі з мэтай кіравання (зніжэння) рызык перадумоў стварэння сітуацый, якія прыводзяць да няшчасных выпадкаў;
    \item забеспячэнне тэхнічна абгрунтаваных умоў эксплуатацыі сістэм вентыляцыі і кандыцыянавання, а таксама абследаванне і кантроль за станам апорных канструкцый будынкаў.
\end{enumerate}

Кіраўнікі дэпартаментаў арганізацыі павінны быць азнаёмлены з загадам аб зацвярджэнні плана мерапрыемстваў па АП, пры гэтым кантроль за выкананнем загаду ускладаецца на дырэктара адміністрацыйнага дэпартамента.

Чаканая сацыяльная эфектыўнасць ад увядзення мерапрыемстваў па АП:
\begin{enumerate}
    \item зніжэнне імавернасці ўзнікнення траўманебяспечных сiтуацый;
    \item падтрыманне умоў працы на працоўных месцах арганізацыі ў адпаведнасці з санітарна-гігіенічнымі нормамі;
    \item паляпшэнне ўмоў навучання, правядзення інструктажоў у арганізацыі;
    \item фарміраванне нарматыўнай базы ў арганізацыі;
    \item зніжэнне рызык ўзнікнення агульных захворванняў;
    \item маніторынг умоў працы на працоўных месцах арганізацыі.
\end{enumerate}

Інжынер па ахове працы, у адпаведнасці са сваімі паўнамоцтвамі, мае права:
\begin{enumerate}
    \item праводзіць праверкі выканання патрабаванняў па АП, стану ўмоў працы;
    \item запытваць і атрымліваць неабходную інфармацыю па пытаннях АП, патрабаваць пісьмовыя тлумачэнні ад службовых асоб, якія працуюць і якія дапусцілі парушэнні патрабаванняў па АП;
    \item прыпыняць ва ўстаноўленым заканадаўствам парадку эксплуатацыю абсталявання, інструмента, прыстасаванняў, транспартных сродкаў, выкананне работ (аказанне паслуг) пры выяўленні парушэнняў, якія ствараюць пагрозу для жыцця або здароўя супрацоўнікаў і навакольных, да іх ліквідацыі;
    \item уносіць прапановы па паляпшэнню ўмоў працы супрацоўнікаў, папярэджваючы вытворчы траўматызм і прафесійныя захворванні.
\end{enumerate}

Такім чынам, асноўнымі задачамі службы аховы працы ў арганізацыі з'яўляюцца:
\begin{enumerate}
    \item арганізацыя навучання і праверкі ведаў супрацоўнікаў па пытаннях аховы працы;
    \item арганізацыя і каардынацыя мерапрыемстваў па ахове працы ў арганізацыі;
    \item кантроль за выкананнем патрабаванняў па АП;
    \item удасканаленне ўмоў працы і правядзенне прафілактычных работ па папярэджанні вытворчага траўматызму, прафесійных і вытворча абумоўленых захворванняў.
\end{enumerate}

\subsection{Ідэнтыфікацыя і аналіз шкодных і небяспечных фактараў}

Арганізацыя «ЭПАМ Сістэмз» валодае мноствам структурных падраздзяленняў, сярод якіх: некалькі дэпартаментаў па распрацоўцы праграмнага забеспячэння, дэпартамент па інфармацыйным забеспячэнні, аддзяленне развіцця сeрвісаў, адміністрацыйны дэпартамент, аддзяленне кіравання персаналам, тэхнічнае аддзяленне, фінансавы дэпартамент і іншыя.

У дадзеным раздзеле будзе праведзена ідэнтыфікацыя шкодных і небяспечных вытворчых фактараў для дэпартамента па распрацоўцы праграмнага забеспячэння.

Дадзены дэпартамент быў абраны, так як распрацоўка праграмнага забеспячэння
з'яўляецца галоўнай функцыяй арганізацыі, а значыць большая частка супрацоўнікаў належыць да дазенага дэпартамента.

Шкодны вытворчы фактар --- гэта фактар, які можа прывесці да розных захворванняў або пагоршыць ўжо існуючыя захворванні.

Небяспечны вытворчы фактар --- гэта фактар, які можа прывесці да траўмы, у тым ліку смяротнай.

Класіфікацыя небяспечных і шкодных вытворчых фактараў па прыродзе ўплыву на здароўе чалавека падрадзяляюцца на:
\begin{enumerate}
    \item фізічныя;
    \item хімічныя;
    \item біялагічныя;
    \item псіхафізіалагічныя.
\end{enumerate}

Наяўнасць фізічных фактараў галоўным чынам абумоўлена выкананнем работ на персанальным камп'ютары пры стварэнні праграмнага забеспячэння. Фізічныя фактары пры працы за камп'ютарам: павышаны ўзровень электрамагнітных выпраменьванняў, падвышаная напружанасць электрычнага поля, падвышанае значэнне напругі ў электрычным ланцугу, замыканне, павышаны ўзровень статычнай электрычнасці, адсутнасць або недахоп натуральнага святла, недастатковая штучная асветленасць працоўнай зоны, падвышаная яркасць святла, падвышаная кантраснасць, манатоннасць працоўнага працэсу.

Галоўным псіхалагічным фактарам у дэпартаменце па распрацоўцы праграмнага забеспячэння з'яўляецца разумовае напружанне, якое можа быць выклікана манатоннасцю працы, невыкананнем рэжыму працы і адпачынку, пастаянная высокая канцэнтрацыя ўвагі, эмацыйная і псіхалагічная нагрузка.

У дэпартаменце па распрацоўцы праграмнага забеспячэння адсутнічаюць непаспрэдныя хімічныя і біялагічныя шкодныя і небяспечныя вытворчыя фактары,
аднак у дэпартаменце па распрацоўцы праграмнага забеспячэння могуць
з'явіцца другасныя біялагічныя шкодныя і небяспечныя фактары (напрыклад, пандэмія інфекцыі COVID-19).

\subsubsection{Арганізацыйныя, тэхналагічныя і іныя рашэнні для
ўстаранення небяспечных і шкодных фактараў}

Небяспечныя і шкодныя вытворчыя фактары становяцца такімі і прыводзяць да парушэння здароўя спецыялістаў, калі іх велічыні перавышаюць гранічна дапушчальныя велічыні, зацверджаныя ва ўстаноўленым парадку.

Згодна з пунктам 8 санітарных норм і правілаў «Гігіенічная класіфікацыя ўмоў працы», зацверджаных пастановай Міністэрства аховы здароўя РБ ад 2012/12/28 № 211, умовы працы дзеляцца на 4 класы:
\begin{enumerate}
    \item 1 клас --- аптымальныя ўмовы працы;
    \item 2 клас --- дапушчальныя ўмовы працы;
    \item 3 клас --- шкодныя ўмовы працы;
    \item 4 клас --- небяспечныя ўмовы працы.
\end{enumerate}

У табліцах \ref{table: healthy1} і \ref{table: healthy2} прыведзены гранічна дапушчальныя ўзроўні электрамагнітных палёў ад ВДТ, ЭВМ і перыферыйных прылад у адпаведнасці з гігіенічным нарматывам «Гранічна дапушчальныя ўзроўні нарміруемых параметраў пры працы з відэадысплейнымі тэрміналамі і электронна-вылічальнымі машынамі».

\begin{table}[htp]
    \caption{Гранічна дапушчальныя ўзроўні электрамагнітных палёў ад ВДТ і ЭВМ}
    \begin{tabularx}{\textwidth}{ | >{\centering\arraybackslash}X
                                  | >{\centering\arraybackslash}X | }
    \hline
        Назва параметра & Гранічна дапушчальныя ўзроўні \\
    \hline
        Напрыжанасць электрычнага поля
        ў дыяпазоне частот & \\
    \hline
        5 Гц - 2 кГц & Не больш за 25 В/м \\
    \hline
        2-400 кГц & Не больш за 2,5 В/м \\
    \hline
        Шчыльнасць магнітнага патоку магнітнага поля
        ў дыяпазоне частот & \\
    \hline
        5 Гц - 2 кГц & Не больш за 250 нТл \\
    \hline
        2-400 кГц & Не больш за 25 нТл \\
    \hline
        Напружанасць электрастатычнага поля & Не больш за 15 кВ/м \\
    \hline
    \end{tabularx}
    \label{table: healthy1}
\end{table}

\vspace{-\baselineskip}

\begin{table}[htp]
    \caption{Гранічна дапушчальныя ўзроўні электрамагнітных палёў ад перыферыйных прылад}
    \begin{tabularx}{\textwidth}{ | >{\centering\arraybackslash}X
                                  | >{\centering\arraybackslash}X
                                  | >{\centering\arraybackslash}X
                                  | >{\centering\arraybackslash}X
                                  | >{\centering\arraybackslash}X
                                  | >{\centering\arraybackslash}X | }
    \hline
        Дыяпазон частот & 0,3 - 300 кГц &
        0,3 - 3 МГц & 3 - 30 МГц & 30 - 300 МГц & 0,3 - 300 ГГц\\
    \hline
        Гранічна дапушчальныя ўзроўні & 25 В/м &
        15 В/м & 10 В/м & 3 В/м & 10 мкВт/кв. см \\
    \hline
    \end{tabularx}
    \label{table: healthy2}
\end{table}

У табліцы \ref{table: healthy3} прыведзены гранічна дапушчальныя ўзроўні інтэнсіўнасці выпраменьвання з боку экрана ВДТ і ЭВМ. У табліцы
\ref{table: healthy4} прыведзены дапушчальныя візуальныя эрганамічныя параметры прылад адлюстравання ВДТ і ЭВМ.

\begin{table}[htp]
    \caption{Гранічна дапушчальныя ўзроўні інтэнсіўнасці выпраменьвання
    з боку экранаў ВДТ і ЭВМ}
    \begin{tabularx}{\textwidth}{ | >{\centering\arraybackslash}X
                                  | >{\centering\arraybackslash}X
                                  | >{\centering\arraybackslash}X
                                  | >{\centering\arraybackslash}X | }
    \hline
        Дыяпазон даўжыні хваль & 200 - 280 нм &
        280 - 315 нм & 315 - 400 нм \\
    \hline
        Гранічна дапушчальныя ўзроўні & Не дапушчальны &
        0,0001 Вт/кв. м. & 0,1 Вт/кв. м. \\
    \hline
    \end{tabularx}
    \label{table: healthy3}
\end{table}

\clearpage

\begin{table}[htp]
    \caption{Дапушчальныя візуальныя эрганамічныя параметры прылад
    адлюстравання ВДТ і ЭВМ}
    \begin{tabularx}{\textwidth}{ | >{\centering\arraybackslash}X
                                  | >{\centering\arraybackslash}X | }
    \hline
        Параметры & Дапушчальныя значэнні \\
    \hline
        Яркасць белага поля & Не менш за 35 кд/кв. м \\
    \hline
        Нераўнамернасць яркасці працоўнага поля & Не больш за +/-20\% \\
    \hline
         Кантрастнасць (манахромны рэжым) & Не менш за 3:1 \\
    \hline
        Часавая нестабільнасць выявы & Не павінна фіксавацца \\
    \hline
        Частата абнаўлення выявы (дыскрэтны экран) & Не менш за 60 Гц \\
    \hline
    \end{tabularx}
    \label{table: healthy4}
\end{table}
