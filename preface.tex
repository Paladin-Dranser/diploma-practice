\sectionWithoutNumber{\prefacename}

У нашыя дні, канцэпцыя SDN (Software Defined Network або праграмна-вызначаная сетка)
імкліва заваёўвае свет сеткавых тэхналогій.

Хуткі рост аб'ёму трафіку прыводзіць да росту інфраструктуры, якая
дазволіць яго апрацаваць. Гэта ў сваю чаргу прыводзіць да таго, што кіраванне
любымі сеткамі становіцца грувасткім, малаэфектыўным і складаным.
А падтрымка інфраструктуры ў сучасным стане становіцца складанай эканамічнай задачай,
так як кожны сеткавы элемент павінен мець усё большую вылічальную здольнасць, каб
апрацоўваць не толькі перадачу трафіка, аднак і будаваць тапалогію сеткі ў рэальным часе.

У працах [1, 2, 3, 4, 5] прапанаваны рашэнні па забеспячэнню механізма
маршрутызацыі ў SDN з улікам выканання патрабаванняў да якасці абслугоўвання
(QoS) для максімальна магчымай колькасці патокаў і ва ўмовах зменлівай нагрузкі.
На аснове высноваў, атрыманых у [4] і [5], будзе разгледжана гібрыдная мадэль
SDN на аснове пратакола маршрутызацыі EIGRP.
Дадзены пратакол можа выкарыстоўваць 5 крытэрыяў для вызначэнян метрыкі маршрута.
Па ўмаўчанні выкарыстоўваюцца толькі прапускная здольнасць канала і затрымка перадачы,
што не дазваляе нагрузкі канала ўплываць на пабудову тапалогіі сеткі.
Дадзенае абмежаванне было ўведзеная з прычыны таго, што паказальнік нагрузкі інтэрфейса
змяняецца ў рэальным часе і можа прывесці да пастаяннага пералічэння тапалогіі сеткі, што
вядзе да высокай нагрузкі на працэсар маршрутызатара і пагаршэння якасці абслугоўвання.

Для таго, каб пазбегнуць вышэй апісаны недахоп, прыняцце рашэння наконт змянення тапалогіі сеткі
будзе вынесены на асобны элемент, на якім будзе вылічвацца змяненне нагрузкі канала адносна
папярэдняга значэння.

Гэта дазволіць паменшыць нагрузку на працэсар маршрутызатара і паменшыць частату змянення
тапалогіі, пры гэтым дазволіўшы сетцы адаптаваць тапалогіі сеткі адносна сеткавых параметраў
у рэальным часе.

\clearpage
