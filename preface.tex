\sectionWithoutNumber{\prefacename}

У нашыя дні, канцэпцыя SDN (Software Defined Network або праграмна-вызначаная сетка)
імкліва заваёўвае свет сеткавых тэхналогій.

Хуткі рост аб'ёму трафіку прыводзіць да росту інфраструктуры, якая
дазволіць яго апрацаваць. Гэта ў сваю чаргу прыводзіць да таго, што кіраванне
любымі сеткамі становіцца грувасткім, малаэфектыўным і складаным.
А падтрымка інфраструктуры ў сучасным стане становіцца складанай эканамічнай задачай,
так як кожны сеткавы элемент павінен мець усё большую вылічальную здольнасць, каб
апрацоўваць не толькі перадачу трафіка, аднак і будаваць тапалогію сеткі ў рэальным часе.

У дадзенай рабоце будзе разгледжана гібрыдная мадэль
SDN на аснове пратакола маршрутызацыі EIGRP.
Дадзены пратакол можа выкарыстоўваць 5 крытэрыяў для вызначэння метрыкі маршрута.
Па ўмаўчанні выкарыстоўваюцца толькі прапускная здольнасць канала і затрымка перадачы,
што не дазваляе нагрузкі канала ўплываць на пабудову тапалогіі сеткі.
Дадзенае абмежаванне было ўведзеная з прычыны таго, што паказальнік нагрузкі інтэрфейса
змяняецца ў рэальным часе і можа прывесці да пастаяннага пералічэння тапалогіі сеткі, што
вядзе да высокай нагрузкі на працэсар маршрутызатара і пагаршэння якасці абслугоўвання.

Для таго, каб пазбегнуць вышэй апісаны недахоп, прыняцце рашэння наконт змянення тапалогіі сеткі
будзе вынесены на асобны элемент, на якім будзе вылічвацца змяненне нагрузкі канала адносна
папярэдняга значэння.

Гэта дазволіць паменшыць нагрузку на працэсар маршрутызатара і паменшыць частату змянення
тапалогіі, пры гэтым дазволіўшы маршрутызатару адаптаваць тапалогію сеткі адносна сеткавых параметраў
у рэальным часе.

Прадметам даследвання ў дыпломнай працы з'яўляецца размеркаванне патокаў трафіка ў гібрыдных сетках SDN на базе EIGRP пратакола.

Мэта дыпломай працы -- праграмна змадэліраваць размеркаванне
патокаў трафіку ў сетках SDN гібрыднай мадэлі на базе распрацаванага алгарытму пераразліку метрыкі
і тапалогіі на сеткавых устройствах для EIGRP пратакола ў рэальным часе
ў залежнасці ад нагрузкі на інтэрфейсах устройства.

Актуальнасць тэмы дыпломнай працы прадыктавана неабходнасцю
змянення падыходаў кіравання патокамі трафіка ў інфакамунікацыйных сетках
у сувязі з пастаянным павелічэннем аб'ёму інфармацыі, якую неабходна
перадаваць у гэтых сетках.

Перадумовы для даследванняў і распрацовак новых падыходаў пры пабудове
SDN сетак гібрыднай мадэлі сфармуляваны немагчымасцю хутка замяніць
інфраструктуру традыцыйных сетак новым пакаленнем SDN устройств у сувязі маштабаў неабходных
работ і эканамічных выдаткаў. У той час як гібрыдная мадэль SDN сетак
прадастаўляе магчымасць пры нязначных змяненнях інфраструктуры традыцыйных сетак атрымаць перавагі ў кіраванні і падтрымцы сеткі, уласцівыя праграмна-вызначаным сеткам.


\clearpage
